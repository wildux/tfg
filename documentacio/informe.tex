\documentclass[a4paper,12pt]{report}
\usepackage[utf8]{inputenc}
\usepackage{graphicx}
\usepackage[catalan]{babel}
\usepackage[backend=bibtex,sorting=none]{biblatex}
\addbibresource{references.bib}
\usepackage[headheight=35pt]{geometry}
\usepackage[bottom]{footmisc}
\usepackage{float}

\usepackage{fancyhdr}
\pagestyle{fancy}
\fancyhead{}
\fancyhead[RO,RE]{\leftmark}
\fancyhead[LO,LE]{\includegraphics[width=0.3\textwidth]{images/fib-upc-logo}}
\fancyfoot{}
\fancyfoot[CO,CE]{\thepage}

\usepackage{titlesec}
\titleformat{\chapter}[hang] 
{\normalfont\huge\bfseries}{\thechapter.}{1em}{} 

\usepackage{multirow}
\usepackage{array,booktabs}
\usepackage{bookmark}
\usepackage[toc,page]{appendix}
\usepackage{rotating}
\usepackage[gen]{eurosym}
\DeclareUnicodeCharacter{20AC}{\euro{}}
\usepackage[table]{xcolor}
\usepackage{hyperref}


%%%%%%%%%%%%%
%PYTHON CODE
%%%%%%%%%%%%%

\DeclareFixedFont{\ttb}{T1}{txtt}{bx}{n}{9} % for bold
\DeclareFixedFont{\ttm}{T1}{txtt}{m}{n}{9}  % for normal
% Defining colors
\usepackage{color}
\definecolor{deepblue}{rgb}{0,0,0.5}
\definecolor{deepred}{rgb}{0.6,0,0}
\definecolor{deepgreen}{rgb}{0,0.5,0}

\definecolor{pblue}{rgb}{0.13,0.13,1}
\definecolor{pgreen}{rgb}{0,0.5,0}
\definecolor{pred}{rgb}{0.9,0,0}
\definecolor{pgrey}{rgb}{0.46,0.45,0.48}


\usepackage{listings}

% Python style for highlighting
\newcommand\pythonstyle{\lstset{
  language=Python,
  backgroundcolor=\color{white}, %%%%%%%
  basicstyle=\ttm,
  otherkeywords={self},            
  keywordstyle=\ttb\color{deepblue},
  emph={MyClass,__init__},          
  emphstyle=\ttb\color{deepred},    
  stringstyle=\color{deepgreen},
  commentstyle=\color{red},  %%%%%%%%
  frame=tb,
  tabsize=2,                         
  showstringspaces=false            
}}

% Python environment
\lstnewenvironment{python}[1][]
{
\pythonstyle
\lstset{#1}
}
{}

% Python style for highlighting
\newcommand\javastyle{\lstset{
  language=Java,
  showspaces=false,
  showtabs=false,
  breaklines=true,
  showstringspaces=false,
  breakatwhitespace=true,
  commentstyle=\color{pgreen},
  keywordstyle=\color{pblue},
  stringstyle=\color{pred},
  basicstyle=\ttfamily,
  moredelim=[il][\textcolor{pgrey}]{$$},
  moredelim=[is][\textcolor{pgrey}]{\%\%}{\%\%}
}}

% Java environment
\lstnewenvironment{java}[1][]
{
\javastyle
\lstset{#1}
}
{}


%%%%%%%%%
%Apèndix
%%%%%%%%%

\renewcommand\appendixtocname{Apèndix}
\renewcommand\appendixpagename{Apèndix}


%%%%%%%%%
%DOCUMENT
%%%%%%%%%

\begin{document}

	\newgeometry{bottom=2.5cm}

	\begin{titlepage}
		\begin{center}
			\vspace*{1cm}

			\line(1,0){400}\\
			%\noindent\hrulefill\\
			\vspace{0.3cm}
			\Huge
			\textbf{Sistema d'autolocalització per a robots mòbils mitjançant tècniques de visió per computador}\\
			%\noindent\hrulefill
			\line(1,0){400}

			\vspace{1.0cm}
			\Large
			Joan Rodas Cusidó\\
			10-12-2016

			\vfill
			\LARGE
			\textit{Treball final de grau en eng. informàtica\\
			Tecnologies de la informació}\\
			\vspace{0.3cm}
			\normalsize{Informe de seguiment}

			\vspace{2cm}

			\includegraphics[width=0.2\textwidth]{images/logo}
			%\includegraphics[width=0.7\textwidth]{images/fib-upc-logo}
			
			%\vspace{1cm}
			\vspace{0.5cm}

			\Large
			Facultat d'Informàtica de Barcelona\\
			Universitat Politècnica de Catalunya\\
			%\vspace{0.5cm}
			Director: Joan Climent Vilaró (ESAII)
		\end{center}
	\end{titlepage}

	\restoregeometry
	\newgeometry{bottom=4.5cm}
	%\newgeometry{bottom=4.0cm}
	\setcounter{page}{2}

%	\chapter*{Resum}
%		Aquest treball final de grau... S'utilitzarà la biblioteca OpenCV per utilitzar diversos algorismes de visió per computador que ens permetran crear un sistema d'autolocalització per a robots.\\\\
	\textit{This final degree project...}

%	\chapter*{Agraïments}
%	En primer lloc, m'agradaria donar les gràcies al director del projecte Joan Climent, per donar-me l'oportunitat de realitzar aquest treball de final de grau.\\\\
També m'agradaria donar les gràcies al meu amic Arnau, que em va fer descobrir el món de GNU/Linux i el programari lliure ja fa anys. Això va fer que m'interesses per la informàtica i possiblement va ser un
factor important a l'hora d'escollir la carrera anys després. I evidentement, també vull donar les gràcies als meus pares i ala meva familia per tot el seu suport.


	\tableofcontents
	\chapter{Introducció i abast}
	\textit{Aquest projecte es desenvolupa com a treball final de grau dels estudis de grau en enginyeria informàtica, de l'especialitat en tecnologies de la informació.
Es tracta d'un projecte de modalitat A, realitzat a la Facultat d'Informàtica de Barcelona (Universitat Politècnica de Catalunya) i proposat pel director Joan Climent,
del departament d'ESAII (Enginyeria de Sistemes, Automàtica i Informàtica Industrial).}\\\\\\\\
Els avenços tecnològics dels últims anys, han millorat la capacitat de les màquines per extreure informació i resoldre problemes de manera autònoma, imitant cada vegada millor
el comportament humà. En aquest treball, es treballarà la visió per computador aplicada a un problema de robòtica.\\\\
El primer capítol serveix com a introducció del projecte, on s'explica l'abast, objectiu, motivació i estat de l'art de les tecnologies a tractar. En el segon capítol es detallen els recursos utilitzats
per realitzar el treball.\\\\
Els capítols 3 a 6 conformen el treball principal. Al tercer capítol s'explica el disseny i l'aquitectura del sistema desenvolupat. Al cuart, les tecniques de visió utilitzades, amb la seva implementació
detallada al cinqué capítol. Al capítol 6 s'explicarán els experiments realitzats i els resultats obtinguts.\\\\
Al capítol 7 podeu trobar un anàlisi de la gestió econòmica del projecte, on es detallen els costos humans, de programari, de maquinari, indirectes i possibles imprevistos. I al vuité capítol
es presenta l'informe de sostenibilitat. Per acabar, hi haurà les conclusions del projecte, on es valorarà l'aportació del projecte a nivell personal i si s'han aconseguit els objectius inicials proposats.

%\section{Context}
%	Aquest projecte es desenvolupa com a treball final de grau dels estudis de grau en enginyeria informàtica, de l'especialitat en tecnologies de la informació.
%	Es tracta d'un projecte de modalitat A, realitzat a la Facultat d'Informàtica de Barcelona (Universitat Politècnica de Catalunya) i proposat pel director Joan Climent,
%	del departament d'ESAII (Enginyeria de Sistemes, Automàtica i Informàtica Industrial).
\section{Descripció del problema}
	El treball pretén resoldre un problema d'autolocalització de robots mòbils en un entorn variable, de tal manera que el robot sigui capaç de desplaçar-se d'un punt inicial a un punt final escollit per
	l'usuari. Per fer això, s'utilitzaran diverses tècniques de visió per ordinador.
\section{Motivació}
	Visió per computador i Robòtica van ser sense cap mena de dubte dos de les assignatures més interessants que he cursat a la universitat, així que quan vaig veure la oferta del projecte vaig pensar que
	seria una bona idea per profunditzar els meus coneixements sobre la materia.
\section{Actors implicats}
	En aquesta secció es descriuen els actors implicats del projecte, és a dir, totes aquelles persones que es veuran beneficiades directa o indirectament amb la realització d'aquest.\\
	\begin{itemize}
		\item \textbf{Autor/Desenvolupador:} És el màxim responsable del projecte. En tractar-se d'un treball final de grau, l'autor del projecte serà també el màxim beneficiari, ja que la realització d'aquest li permetrà acabar la carrera d'enginyeria informàtica.
		\item \textbf{Usuaris:} Qualsevol persona qui ho desitgi, tindrà accés a tots els codis desenvolupats durant el projecte, ja que es llançaran sota una llicència de programari lliure que permetrà veure i adaptar el codi a les necessitats d'altres usuaris.
		\item \textbf{Altres beneficiaris:} Qualsevol empresa o institució interessada podrà utilitzar el sistema desenvolupat i adaptar-lo a les seves necessitats, com podria ser per exemple un sistema de transport d'equipatge basat en robots.
	\end{itemize}
\newpage
\section{Estat de l'art}
	\subsection{Visió per computador}
		La visió per computador\cite{Szeliski} és una ciència que té com a objectiu dotar les màquines o ordinadors de la capacitat de ``veure''.
		Es basa en l'extracció i anàlisi de dades obtingudes a partir d'imatges.\\\\
		Algunes de les aplicacions de la visió per computador són:\\
		\begin{itemize}
			\item Vehicles autònoms
			\item Realitat augmentada
			\item Reconeixement facial
			\item Restauració d'imatges
			\item Inspecció industrial 
			\item Robòtica\\
		\end{itemize}
		En aquest treball, ens interessa utilitzar la visió per computador en el camp de la robòtica, per aconseguir guiar a un robot mòbil cap a un objectiu
		determinat basant-se en la detecció d'un punt o regió en una imatge.
		\subsubsection{Nous algorismes}
			En els darrers anys, han aparegut nous algorismes d'obtenció de punts i extracció de característiques que suposen una alternativa als clàssics SIFT\cite{SIFT}
			( Scale Invariant Feature Transform) i SURF\cite{SURF} (Speeded-Up Robust Features). Alguns d'aquests algorismes són BinBoost\cite{Trzcinski13a} o un dels més recents:
			LATCH\cite{LeviHassner2016LATCH}\\\\
			En aquest projecte s'analitzarà si es adequat emprar algun d'aquests algorismes en la implementació del sistema d'autolocalització. 
\newpage
	\subsection{Robòtica}
		La robòtica és un camp de la tecnologia que estudia el disseny i la construcció de robots.\\\\
		Que és, doncs, un robot? Al llarg de la historia, s'han donat diverses definicions del concepte de robot, sense existir encara una definició exacta acceptada per tothom. I a mesura que passa el temps,
		cada vegada resulta més complicat determinar si una màquina és o no un robot. Per no complicar-nos massa, entendrem com a robot una màquina programable capaç de realitzar una sèrie de
		tasques concretes interactuant amb l'entorn, ja sigui de manera automàtica o dirigida.\\\\
		Existeixen diversos tipus de robots, podent fer una classificació senzilla segons la seva arquitectura: robots mòbils, poliarticulats (industrials, mèdics, etc.), humanoides, 
		zoomòrfics\footnote{\textbf{Robots zoomòrfics:} Robots que imiten característiques pròpies de determinats animals.} i híbrids.\\\\
		Els robots mòbils, que són els que ens interessen per aquest projecte, acostumen a tenir una sèrie de sensors i dispositius per permetre'n el desplaçament, la localització, esquivar obstacles i
		realitzar tasques concretes. Alguns exemples de sensors utilitzats per robots mòbils són:\\
		\begin{itemize}
			\item Odometria: S'utilitza la informació obtinguda amb sensors de moviment (\textit{encoders} a les rodes, per exemple) per estimar la posició del robot respecte a la inicial.
			\item GPS (Global Positioning System): Es determina la ubicació del robot amb la xarxa de satèl·lits.
			\item Sensors de contacte: Permeten detectar si el robot està en contacte amb un altre objecte.
			\item Sensors d'ultrasons: Detecten objectes mitjançant ones ultrasòniques.
			\item Acceleròmetre: Determina l'acceleració del robot quan es mou. 
			\item Càmera: Permet capturar imatges de l'entorn.\\
		\end{itemize}
		En el nostre cas, només ens interessaran les dades obtingudes a través d'una càmera, és a dir, les imatges. El treball no se centrarà per tant en la part robòtica del sistema, i no es tindran en compte
		els sensors i algorismes necessaris per poder moure el robot.\\\\
		En cas d'aplicar el sistema desenvolupat en robots en un futur, aleshores s'hauran de tenir en compte altres sensors per permetre el moviment
		de la màquina i arribar a la destinació evitant obstacles.
\section{Objectius}
	L'objectiu principal del projecte consisteix a dissenyar i desenvolupar un sistema d'autolocalització per a robots mòbils.\\\\
	Aquest sistema estarà basat en tècniques de visió per computador i consistirà, bàsicament, a comparar dues imatges (una global i una altra capturada pel robot)
	i localitzar un punt o regió seleccionat per l'usuari.\\\\	
	Per arribar a aquest objectiu, es dividirà el treball en diverses fases:\\
	\begin{itemize}
		\item Estudi dels diferents algorismes de visió existents
		\item Obtenció de \textit{keypoints} en una imatge
		\item Extracció de característiques
		\item \textit{Matching} de dues imatges
	\end{itemize}
\section{Requeriments}
	El sistema d'autolocalització implementat ha de complir amb una sèrie de requeriments mínims presentats a continuació:\\
	\begin{itemize}
		\item L'usuari ha de poder seleccionar un punt o regió d'interès en una imatge donada.
		\item El sistema ha de ser capaç d'adaptar-se mínimament a diverses condicions de l'entorn (canvis de lluminositat, perspectiva, etc.).
	\end{itemize}
\section{Obstacles}
	Durant la planificació i realització del treball, s'hauran de tenir en compte els possibles obstacles que es trobaran. A continuació es detallen alguns dels problemes que es podran trobar.
	\subsubsection{Noves eines}
		Un dels principals obstacles serà el fet de treballar amb noves eines i algorismes. Per tal d'evitar problemes en aquest aspecte, caldrà fer una planificació acurada i documentar-se apropiadament.
		També serà important mantenir una bona comunicació amb el tutor en tot moment, per poder resoldre possibles dubtes referents als algorismes.
	\subsubsection{Calendari}
		Un altre obstacle important serà la falta de temps, ja que està previst realitzar el projecte en el transcurs d'un quadrimestre. Gestionar correctament el temps serà clau per aconseguir
		finalitzar el projecte sense problemes. Per tant, s'haurà de fer una planificació el més realista possible i escollir una metodologia de treball adequada i flexible.
	\subsubsection{Errors de programació}
		Com a qualsevol projecte on s'ha de programar, el codi serà una font important d'errors. Per això, caldrà realitzar diverses proves cada vegada que es realitzi una modificació en el codi
		o s'implementi una nova funcionalitat.
	\subsubsection{Condicions variables en les imatges}
		Les imatges capturades a través d'una càmera no presentaran sempre les mateixes condicions. La lluminositat, perspectiva o resolució de la imatge 
		influiran a l'hora de processar les imatges i comparar-les.\\\\
		Per intentar minimitzar aquests efectes, s'analitzaran diversos algorismes d'obtenció de punts i extracció de característiques. 
		També s'estudiarà si és necessari realitzar un preprocessament o filtratge de les imatges abans d'aplicar els algorismes.
\section{Ampliacions}
	Encara que el calendari és força estricte i no hi ha gaire marge d'ampliació, es podria estendre el projecte amb les següents ampliacions:\\
	\begin{itemize}
		\item Anàlisi del rendiment d'algorismes alternatius per l'obtenció de punts i característiques de les imatges.
		\item Creació d'una aplicació d'Android que permeti seleccionar un punt o regió d'una imatge.
		\item Execució del codi del sistema via servidor web, utilitzant les dades enviades per l'aplicació d'Android.
	\end{itemize}
\section{Metodologia}
	Per aquest projecte, s'utilitzarà una metodologia de treball àgil amb cicles de desenvolupament curts.
	Com que només hi ha un desenvolupador, no s'utilitzaran exactament les metodologies Scrum o XP\cite{Pxp} (Extreme Programming),
	però sí que s'aplicaran moltes de les pràctiques pròpies d'aquestes dues metodologies (proves, simplicitat, refacció de codi, etc.).
	Això ens donarà més flexibilitat a l'hora de fer canvis i adaptar-nos a una nova planificació.\\\\
	Es començarà treballant amb imatges de prova (casos senzills) i algorismes coneguts com ara Harris\cite{Harris} i SIFT. Més endavant, s'aniran introduint modificacions en el codi per intentar aconseguir un
	sistema capaç de funcionar amb fotografies ``reals'' i es provaran altres algorismes de visió per computador.\\\\
	Per altra banda, s'utilitzarà el mètode en cascada per la realització del curs de GEP.
\section{Eines de desenvolupament}
	El codi del projecte es desenvoluparà amb python i s'utilitzaran, sempre que sigui possible, eines de programari lliure o de codi obert.\\\\
	En cas de crear una aplicació per a dispositius Android, es realitzarà mitjançant Android Studio (Java).
	\subsection{OpenCV}
		Per tal d'utilitzar algorismes de visió per computador en el codi amb relativa facilitat, s'utilitzarà la biblioteca de codi obert OpenCV\cite{OpenCV} (Open Source Computer Vision Library),
		disponible per a python. La versió emprada serà la 3.1.\\\\
		En concret, hi haurà tres passos indispensables que faran ús d'aquesta biblioteca:\\
		\begin{itemize}
			\item {Obtenció de punts en una imatge}
			\item {Extracció de característiques}
			\item \textit{Matching} de dues imatges
		\end{itemize}
\section{Eines de seguiment}
	A continuació es detallen les eines de programari usades per fer el seguiment del treball final de grau:
	\subsubsection{LibreOffice Calc}
		Per fer un seguiment de les hores dedicades al projecte, es crearà un full de càlcul amb les hores diàries dedicades a cada tasca. S'utilitzarà LibreOffice Calc, inclòs en
		la \textit{suite} ofimàtica LibreOffice.
	\subsubsection{Gantt Project}
		Per tal d'organitzar totes les tasques a realitzar i mirar si hi ha desviacions respecte el pla inicial, s'utilitzarà l'eina de \textit{software} lliure 
		Gantt Project\cite{GanttProject}. Aquesta eina ens permetrà realitzar tant un diagrama de Gantt com un diagrama de PERT.
	\subsubsection{Git + Github}
		Tot i que no es tracta d'un projecte col·laboratiu (només hi ha un desenvolupador), s'ha decidit utilitzar el sistema de control de versions Git juntament amb la pàgina web Github. 
		D'aquesta manera es facilitarà treballar amb diverses màquines i portar un control dels canvis realitzats. A més, permetrà compartir el codi amb el director amb facilitat.
	\section{Mètode de validació}
		Es faran validacions parcials durant la realització del projecte, fent proves del sistema amb diverses imatges.
	\subsubsection{Contacte amb el director}
		Hi haurà reunions presencials amb el director, així com comunicació via correu electrònic, per tal de resoldre dubtes i validar la feina realitzada. També es realitzarà una reunió de seguiment abans
		del 19 de decembre, per conèixer l'estat del projecte i poder escollir el torn de lectura.


	\chapter{Planificació i recursos}
	%\newcommand*{\thead}[1]{\multicolumn{1}{|l|}{\bfseries #1}}
\def\arraystretch{1.4}
\definecolor{tableHeader}{RGB}{211, 127, 47}
\definecolor{myOrange}{RGB}{255, 230, 210}
\newcolumntype{x}[1]{>{\centering\arraybackslash\hspace{0pt}}m{#1}}

\section{Planificació temporal}
	\label{sec:planificacio}
	El treball té una duració aproximada de 4 mesos i mig, des de setembre fins a principis de gener. La càrrega total serà d'unes 450 hores, corresponents a 18 crèdits ECTS.
	La dedicació setmanal estimada serà d'unes 25 hores.\\\\
	Es dividirà el projecte en quatre blocs, descrits a continuació:\\

	\begin{table}[H]
		\begin{center}
			\rowcolors{2}{myOrange}{white}
			\begin{tabular}{p{1.5cm} !{\vrule width -1pt}p{6cm} !{\vrule width -1pt}l !{\vrule width -1pt}l}
			\rowcolor{tableHeader}
			\textbf{Bloc} & \textbf{Descripció} & \textbf{Metodologia} & \textbf{Hores} \\
			Bloc 0 & Preparació de l'entorn & - & 5h \\
			Bloc 1 & Curs de GEP & Cascada & 75h \\
			Bloc 2 & Desenvolupament del projecte & Àgil & 340h \\
			Bloc 3 & Preparació de la defensa & - & 30h \\
			\end{tabular}
		\end{center}
		\caption{Blocs del projecte}
	\end{table}

	\subsection{Bloc 0: Preparació de l'entorn}
		Inicialment, s'instal·larà tot el programari necessari per començar a desenvolupar el projecte i es faran algunes proves bàsiques per familiaritzar-se amb el nou entorn de treball.
		Aquest primer bloc tindrà una durada aproximada de 5 hores.\\\\
		Per poder començar a treballar en el projecte, caldrà instal·lar:\\
		\begin{itemize}
			\item \textbf{Desenvolupament:} Python, OpenCV, Geany i Git.
			\item \textbf{Curs de GEP:} Gantt Project.
			\item \textbf{Documentació:} {\LaTeX} i Zathura.
		\end{itemize}

	\subsection{Bloc 1: Curs de GEP}
		Aquest bloc correspon a la realització del curs de GEP, amb inici el dia 19/09/2016 i finalització el 24/10/2016 (amb una presentació final entre el 7 i l'11 de novembre).
		Té com a dependència el bloc 0.\\\\
		Durant el curs s'entregaran 6 lliurables, detallats a continuació:\\
		\begin{table}[H]
			\begin{center}
				\rowcolors{2}{myOrange}{white}
				\begin{tabular}{p{5cm} !{\vrule width -1pt}x{2.1cm} !{\vrule width -1pt}x{2.4cm} !{\vrule width -1pt}x{1.6cm} !{\vrule width -1pt}x{1.6cm}}
				\rowcolor{tableHeader}
				\textbf{Descripció} & \textbf{Inici} & \textbf{Finalització} & \textbf{Durada} & \textbf{Hores} \\
				Introducció i abast & 19/09/2016 & 27/09/2016 & 9 dies & 16h \\
				Planificació temporal & 28/09/2016 & 03/10/2016 & 6 dies & 9h \\
				Gestió econòmica i sostenibilitat & 04/10/2016 & 10/10/2016 & 7 dies & 10h \\
				Presentació preliminar en vídeo & 11/10/2016 & 17/10/2016 & 7 dies & 11h \\
				Plec de condicions & 18/10/2016 & 24/10/2016 & 7 dies & 13h \\
				Document final + presentació & 18/10/2016 & 24/10/2016 & 7 dies & 16h
				\end{tabular}
			\end{center}
			\caption{Lliurables de GEP}
		\end{table}

	\subsection{Bloc 2: Desenvolupament del projecte}
			El bloc principal consistirà en el desenvolupament del projecte en si mateix: buscar informació, implementar el codi, redactar la memòria, etc.\\\\
			Aquest bloc té com a dependència el bloc 0 i es dividirà en quatre tasques.\\
			\begin{table}[H]
				\begin{center}
					\rowcolors{2}{myOrange}{white}
					\begin{tabular}{p{5cm} !{\vrule width -1pt}x{2.1cm} !{\vrule width -1pt}x{2.4cm} !{\vrule width -1pt}x{1.6cm} !{\vrule width -1pt}x{1.6cm}}
					\rowcolor{tableHeader}
					\textbf{Tasca} & \textbf{Inici} & \textbf{Finalització} & \textbf{Durada} & \textbf{Hores} \\
					Implementació i proves & 13/09/2016 & 19/12/2016 & 98 dies & 240h \\
					Conclusions i resultats & 20/12/2016 & 31/12/2016 & 12 dies & 30h \\
					Ampliacions (opcional) & 01/01/2017 & 09/01/2017 & 9 dies & 30h \\
					Redacció de la memòria & 10/10/2016 & 09/01/2016 & 92 dies & 40h \\
					\end{tabular}
				\end{center}
				\caption{Tasques desenvolupament}
			\end{table}

		\subsubsection{Recerca d'informació, implementació i proves}
			Una part molt important del projecte serà la cerca d'informació i l'estudi de les diverses eines i algorismes a utilitzar (com per exemple OpenCV i les seves funcions).
			Se cercarà informació contínuament i s'aniran fent proves a mesura que s'implementa el codi.\\\\
			La fase d'implementació es dividirà en diverses tasques, que s'aniràn realitzant a mesura que avanci el projecte. Algunes d'aquestes tasques seràn:
			\begin{itemize}
				\item{Obtenció de keypoints (Harris)}
				\item{Extracció de característiques (SIFT)}
				\item{Matching i homografia}
				\item{Altres algorismes (ORB, BRISK, BRIEF)}
				\item{Disseny de l'aplicació d'Android}
				\item{Creació de l'aplicació d'Android}
			\end{itemize}
		\subsubsection{Conclusions i resultats}
			Un cop enllestida la implementació, es procedirà a elaborar les conclusions d'acord amb els resultats obtinguts. Es compararan els resultats obtinguts amb diferents algorismes de visió i es faran
			proves del sistema amb diverses imatges.
		\subsubsection{Ampliacions (opcional)}
			Un cop realitzades les conclusions, en cas de disposar de més temps, es podran fer ampliacions i millores.\\\\
			En cas de patir un retard en la planificació del projecte, s'utilitzarà aquest temps per acabar l'etapa de les conclusions.
		\subsubsection{Redacció de la memòria}
			La memòria s'anirà redactant a mesura que es realitza el projecte. No hi ha per tant cap dependència, encara que es dedicarà més temps en l'etapa final del treball.\\

		\begin{figure}[H]
			\centering
			\includegraphics[width=0.7\textwidth]{images/tasques}
			\caption{PERT - tasques desenvolupament}
		\end{figure}

	\subsection{Bloc 3: Preparació de la defensa}
	En aquest bloc final es revisarà la memòria del projecte i es prepararà la presentació. Està previst dedicar unes 30 hores al bloc, que començarà el dia 10 de gener i acabarà el 22.
	La defensa del projecte es durà a terme entre els dies 23 i 27 de gener.

	\subsection{Diagrames}
		Durant la fase de planificació del projecte, s'han realitzat diversos diagrames. Podeu trobar tots aquests diagrames (Gantt i PERT) a l'apèndix \ref{appendix:diagrames} del treball.

\section{Recursos}
	En aquesta secció s'analitzen els recursos necessaris per a la realització del projecte. %A continuació es detallen els recursos humans, de maquinari i de programari utilizats.
	\subsection{Recursos humans}
		El projecte el realitzarà una sola persona, que haurà d'assumir els rols de cap de projecte, analista, dissenyador, programador i \textit{tester}.
		També es comptarà amb l'ajuda del director del projecte, que assumirà el paper de consultor/supervisor.
	\subsection{Recursos de maquinari}
		Per la realització del projecte no serà necessari adquirir cap mena de maquinari específic. Es podrà utilitzar un ordinador personal per treballar a casa i els ordinadors disponibles a la FIB per
		treballar des de la universitat.\\\\
		Es treballarà principalment amb un ordinador equipat amb un processador AMD FX 6300 hexa-core 3.5GHz, 4GB de RAM i 250GB de disc dur SSD. També s'utilitzarà una càmera o smartphone qualsevol.
	\subsection{Recursos de programari}
	Durant la realització del projecte i el curs de GEP, s'utilitzaran diverses eines de programari, detallades a continuació:\\
	\begin{table}[H]
		\begin{center}
			\rowcolors{2}{myOrange}{white}
			\begin{tabular}{l !{\vrule width -1pt}p{5cm} !{\vrule width -1pt}p{5cm}}
			\rowcolor{tableHeader}
			\textbf{Nom} & \textbf{Tipus} & \textbf{Ús} \\
			Arch Linux & Eina de desenvolupament & Execució del programari \\
			Python & Eina de desenvolupament & Programació \\
			OpenCV & Eina de desenvolupament & Algorismes de VC \\
			Geany & Eina de desenvolupament & Programació del codi \\
			Android Studio & Eina de desenvolupament & Programació del codi \\
			Gimp/Inkscape & Eina de desenvolupament & Retocs i creació d'imatges \\
			\LaTeX & Eina de desenvolupament & Redacció de la memòria \\
			Zathura & Eina de desenvolupament & Visualització de pdf \\
			Gantt Project & Eina de gestió & Creació diagrames de Gantt \\
			LibreOffice Calc & Eina de gestió & Control de les hores \\
			Git + Github & Desenvolupament i gestió & Control de versions \\
			\end{tabular}
		\end{center}
		\caption{Recursos de programari}
		\label{table:programari}
	\end{table}
	
\section{Desviacions i pla d'actuació}
	\subsubsection{Mala planificació [Impacte: baix]}
		Hi haurà reunions amb el director i s'usaran eines de planificació per mirar de corregir la planificació i acabar el projecte a temps. També es reserven unes hores a l'ampliació del treball,
		que es podrien utilitzar en cas que una tasca s'allargués més del previst. Si fos necessari, es podria incrementar una mica la càrrega de treball setmanal.
	\subsubsection{Fallades de maquinari [Impacte: baix]}
		En cas de fallades en l'ordinador principal, no hi hauria cap problema en utilitzar-ne un altre. No hi ha dependències de hardware i es disposa d'altres ordinadors (a casa i a la FIB).
		Tampoc hi hauria una pèrdua de dades important, ja que es treballa amb Github i una còpia local.

%	\chapter{Disseny i arquitectura}
%	\label{sec:Disseny}

En aquest capítol es detalla l'arquitectura del sistema i el disseny de les aplicacions desenvolupades.
\section{Arquitectura del sistema}
	L'arquitectura del sistema està formada per tres parts diferenciades: el client (aplicació web o per \textit{smartphones}), un servidor i el robot. A continuació podeu veure un esquema de l'arquitectura i l'explicació
	de cada una de les parts.\\
	\begin{figure}[H]
		\centering
		\includegraphics[width=0.7\textwidth]{images/arquitectura}
		\captionsource{Arquitectura del sistema}{Madebyoliver i Pixel Buddha}
	\end{figure}
	\vspace{0.05cm}
	\begin{itemize}
		\item{\textbf{Aplicació mòbil/web:} Permet a l'usuari seleccionar una regió en una imatge i l'envia al servidor.}
		\item{\textbf{Servidor:} Rep la selecció de l'aplicació i les imatges del robot. S'encarrega de fer el \textit{matching} per obtenir el punt on s'haurà de desplaçar el robot.}
		\item{\textbf{Robot:} Envia les imatges capturades al servidor i espera rebre les ordres per desplaçar-se i girar.\\}
	\end{itemize}
	El treball se centrarà en la part del servidor i segons el temps disponible es realitzarà la comunicació entre el servidor i l'aplicació per a \textit{smartphones}.

\section{Aplicació de proves}
	Inicialment es desenvoluparà una aplicació per realitzar les diverses proves sense interfície gràfica, que simplement mostrarà el resultat obtingut del \textit{matching}. Més endavant, però, s'inclourà una
	interfície mínima que ens permetrà escollir les imatges a tractar i la regió d'interès (punt de destí).
	\subsubsection{Selecció de les imatges}
		Per seleccionar les imatges s'utilitzarà Tkinter\cite{Tkinter}, una GUI estàndard de Python.
		%\begin{figure}[H]
		%	\centering
		%	\includegraphics[width=0.5\textwidth]{images/fs}
		%	\caption{Selecció de les imatges}
		%\end{figure}

	\subsubsection{Selecció de la regió d'interès}
		Per poder escollir el punt de destí del robot, l'usuari haurà de seleccionar una regió d'interès en una imatge. Això es farà de manera molt senzilla, fent una selecció amb el ratolí.
		Un cop realitzada la selecció, l'usuari tindrà l'opció de refer-la (simplement seleccionant de nou) o d'acceptar-la polsant la tecla c.\\\\
		%\begin{figure}[H]
		%	\centering
		%	\includegraphics[width=0.5\textwidth]{images/selection}
		%	\caption{Selecció de la regió d'interès}
		%\end{figure}
		\begin{figure}[!htb]
			\resizebox{\textwidth}{!}{%
			\includegraphics[height=1cm]{images/fs}
			\includegraphics[height=1cm]{images/selection}}
			\caption{Aplicació de proves}
		\end{figure}
\newpage
\section{Disseny de l'aplicació mòbil}
	La interfície de l'aplicació mòbil i de l'aplicació web ha de ser molt simple. L'usuari només hauria de veure una imatge, d'on podria seleccionar-ne una regió i acceptar la selecció.\\
	%\begin{figure}[H]
	%		\centering
	%		\includegraphics[width=0.6\textwidth]{images/crop}
	%		\caption{App - Selecció de la regió d'interès}
	%\end{figure}
	\begin{figure}[H]
		\resizebox{\textwidth}{!}{%
		\includegraphics[height=1cm]{images/crop}
		\includegraphics[height=1cm]{images/webapp}}
		\caption{App/Webapp - Selecció de la regió d'interès}
	\end{figure}
	\noindent
	\\{}
	També s'hauria d'oferir una opció pels administradors, per tal de poder canviar la imatge a mostrar. El menú principal de l'aplicacio mòbil hauria de tenir les següents opcions:
	%\begin{itemize}
	%	\item{\textbf{Fer una foto}}
	%	\item{\textbf{Seleccionar una imatge de la galeria}}
	%	\item{\textbf{Opcions}}
	%\end{itemize}
	%\begin{figure}[H]
	%	\centering
	%	\includegraphics[width=0.4\textwidth]{images/menu}
	%	\caption{App - Menú}
	%\end{figure}
	\subsubsection{Fer una foto}
		Permet a l'usuari fer una foto amb la càmera del dispositiu, per després pujar-la al servidor.\\
		\begin{figure}[H]
			\centering
			\includegraphics[width=0.6\textwidth]{images/cam}
			\caption{App - Càmera}
		\end{figure}
	\subsubsection{Seleccionar una imatge}
	Si ja es disposa d'una imatge de l'entorn a la memòria del dispositiu, aquesta opció permetria a l'usuari seleccionar-la.
		\begin{figure}[H]
			\centering
			\includegraphics[width=0.4\textwidth]{images/gallery}
			\caption{App - Galeria}
		\end{figure}
	%\subsubsection{Selecció de la regió d'interès}
	%	Desprès de realitzar una captura o seleccionar una imatge del dispositiu, el programa demanarà a l'usuari que seleccioni una regió d'interès (destí) on vol que es desplaçi el robot.
	%	\begin{figure}[H]
	%		\centering
	%		\includegraphics[width=0.6\textwidth]{images/crop}
	%		\caption{App - Selecció de la regió d'interès}
	%	\end{figure}
	\subsubsection{Opcions}
		Des del menú d'opcions, l'usuari hauria de poder posar les dades per connectar-se al robot (adreça IP).\\\\
		També s'haurien de poder escollir els algorismes de visió a utilitzar.
		%I pels usuaris avançats, existeixen opcions per canviar els algorismes de visió per defecte:\\
		%\begin{itemize}
		%	\item{Obtenció de keypoints: Harris, SIFT, SURF, ORB, BRISK, MSER}
		%	\item{Extracció de característiques: SIFT, SURF, ORB, BRISK, LATCH, DAISY\\}
		%\end{itemize}
		
		%\begin{figure}[H]
		%	\centering
		%	\includegraphics[width=0.4\textwidth]{images/options}
		%	\caption{App - Opcions}
		%\end{figure}

%	\chapter{Tècniques de visió usades}
%	En aquesta secció es descriuen les tècniques de visió per ordinador i tractament d'imatges emprades durant la realització del projecte.

\section{Pre-processat digital d'imatges}
	El pre-processament digital en una imatge, consisteix en aplicar diverses tècniques per tal d'aconseguir una imatge d'on poder obtenir la informació que necessitem més facilment. Es tracta
	d'eliminar distorsions o be ressaltar determinades parts de la imatge.\\\\
	Algunes de les tècniques que es poden aplicar i s'han provat durant la realització del projecte són:

	\begin{itemize}
		\item{Suavitzat de la imatge i reducció de soroll: S'han provat filtres senzills com la mediana.}
		\item{Reducció de mides: Reduir la mida de la imatge ens permet millorar el temps d'execució.}
		\item{Escala de grisos: Els píxels de la imatge passen a tenir un valor en el rang 0-255. D'aquesta manera s'aconsegueix reduir el pes de la imatge. Encara que es perd la informació del color, moltes
		vegades pot ser irellevant o fins i tot portar a errors.}
		\item{Equalització de l'histograma: Per tal de millorar el contrast de les imatges, s'ha provat d'utilitzar Clahe local adaptatiu.}
		\item{Operacions morfologiques: Erode, dilate, open, close}
	\end{itemize}
\noindent
	Finalment, s'ha decidit no aplicar cap filtre i utilitzar les imatges tal com són, ja que o bé no feien cap efecte o funcionaven correctament nomes amb determinats tipus d'imatges i empitjoraven d'altres.
	El que si que s'ha fet es reduir la mida i convertir les imatges a escala de grisos.

\newpage
\section{Obtenció de keypoints en una imatge}
	Consisteix en obtenir punts de la imatge amb característiques distintives, que ens puguin ser útils més endavant.\\
	\begin{figure}[H]
		\centering
		\includegraphics[width=0.65\textwidth]{images/RobotKp}
		\caption{Keypoints}
	\end{figure}
	\noindent
	La detecció es pot classificar en:\\
	\begin{itemize}
		\item{Detecció de vores}
		\item{Detecció de cantonades}
		\item{Detecció de regions\\}
	\end{itemize}
	Les principals tècniques d'obtenció de keypoints que s'han utilitzat són: SIFT, HARRIS i ORB.
	També s'han provat altres algorismes com FAST, SURF, STAR, MSER o SHI-TOMASI.

	\subsubsection{HARRIS}
	Harris és un detector que es basa en trobar cantonades. La idea bàsica és que mitjançant una finestra de NxM píxels, es recorre la imatge buscant els punts on hi ha canvis d'intensitat
	en diverses direccions.\\\\
	Segons els canvis d'intensitat de cada píxel, es poden classificar els punts en "flat", vores i cantonades.\\\\
	\begin{figure}[H]
		\centering
		\includegraphics[width=0.7\textwidth]{images/harris}
		\caption{Flat, vora i cantonada}
	\end{figure}
	\noindent
	El problema principal de Harris és que no és invariant a l'escala. Punts considerats cantonades en una escala, podrien convertir-se en vores en una altra.
	Per això, existeixen algunes solucions com Harris-Laplace.\\\\
	En el nostre cas, s'ha optat per aplicar Harris en diverses escales, fent una piràmide de la imatge original.

	\subsubsection{SIFT}
	Localització multi-escala mitjançant una diferencia de gaussianes (DoG), que s'utilitza com a aproximació d'una laplaciana de gaussianes.\\\\
	Es troba l'histograma d'orientacions ponderat en l'escala calculada i es calcula l'orientació dominant.

	\subsubsection{ORB}
	El detector de ORB (Oriented FAST and Rotated BRIEF) utilitza el detector FAST amb modificacions per millorar el rendiment.\\\\
	FAST és un detector de cantonades enfocat a aconseguir els punts de manera molt ràpida, a canvi d'empitjorar una mica l'eficacia. S'agafa la intensitat d'un píxel i es compara amb el conjunt
	de N píxels veins que el rodejen.\\\\
	\begin{figure}[H]
		\centering
		\includegraphics[width=0.55\textwidth]{images/fast}
		\caption{FAST}
	\end{figure}
	\noindent
	Suposant que N = 16 i definint un umbral t, si 12 píxels veins (\sfrac{3}{4} parts) són majors a I\textsubscript{p} + t o bé menors a I\textsubscript{p} - t, és considera que el punt és d'interès.
	El problema principal de FAST és que no té en compte l'orientació.\\\\
	ORB utilitza FAST aplicant les següents millores:\\
	\begin{itemize}
		\item{S'agafen els N millors punts despres d'aplicar la mesura de Harris.}
		\item{Es fa una piràmide per fer multi-escala.}
		\item{S'utilitzen els moments per calcular l'orientació.}
	\end{itemize}

\section{Extracció de característiques}

	L'extracció de característiques és el que ens permetrà comparar els punts obtinguts en les dues imatges.\\
	\begin{itemize}
		\item{Descriptors vectorials: SIFT, SURF}
		\item{Descriptors binaris: ORB, BRISK\\}
	\end{itemize}
	Principalment s'han utilitzat SIFT, ORB i BRISK, encara que també s'han provat d'altres com SURF, LATCH o DAISY.

	\subsubsection{SIFT}
	S'agafa l'histograma d'orientacions al voltant del punt, discretitzant en una finestra de 16x16.
	Es divideix en 16 particions de mida 4x4 i per cada partició es calcula un histograma d'orientacions, en 8 direccions.
	Finalment, es concatenen els histogrames per tal d'obtenir el vector de característiques de dimensió 128.
	\begin{figure}[H]
		\centering
		\includegraphics[width=0.65\textwidth]{images/sift-des}
		\caption{Descriptor SIFT}
	\end{figure}

	\subsubsection{ORB}
	El descriptor d'ORB és una modificació de BRIEF, un descriptor ràpid i senzill basat en strings binaris.\\\\
	Per cada keypoint s'agafen N punts veins i d'aquests s'agafen parells de forma mes o menys aleatoria. Per cada parell es compara la intensitat i es retorna un string binari de mida N amb '1' o '0' segons
	si la intensitat del primer punt és major a la del segon o no. Això ens permet comparar els descriptors amb una simple operació XOR.\\\\
	És un mètode molt ràpid pero no és invariable ni a la rotació ni a l'escala. Per això, ORB intenta solucionar el problema de la rotació girant els patrons en funció de l'angle de la característica.

	\subsubsection{BRISK}
	És un descriptor binari que utilitza un patró de cercles concentrics. S'agafen N punts del patró i per cada parell de punts es compara la intensitat del primer amb el segon. Si el valor del primer es
	major, es posa '1', sino '0'. D'aquesta manera s'obté una cadena d'N caràcters (binari) molt fàcil de comparar a l'hora de fer matching. Utilitizant el mateix patró i sequencia de parells, nomes caldrà
	comparar les cadenes binaries amb la suma del resultat d'una XOR.
	\begin{figure}[H]
		\centering
		\includegraphics[width=0.5\textwidth]{images/brisk}
		\caption{Sampling pattern BRISK}
	\end{figure}


\newpage
\section{Matching de característiques}

	Un cop tenim els keypoints i les característiques dels punts, necessitarem obtenir coincidencies entre els punts de les dues imatges.\\\\
	Bàsicament podem obtenir els aparellaments de dues maneres:\\
	\begin{itemize}	
		\item{Força bruta: Consisteix en provar totes les combinacions possibles per cada punt.}
		\item{Aproximació\\}
	\end{itemize}
	Com que el temps d'execució no es una factor essencial, s'aplicarà el mètode de força bruta, una mica més lent. Pels descriptors binaris s'utilitzarà la distància de Hamming, mentre que pels vectorials
	s'utilitzarà l'euclidiana.\\

	\begin{figure}[H]
		\centering
		\includegraphics[width=0.7\textwidth]{images/matching}
		\caption{Matching}
	\end{figure}
\noindent
	En la imatge anterior podem veure determinats punts on el "match" és clarament erroni. Aquest error es pot minimitzar escollint punts més significatius, característiques més distinctives, aplicant el rati
	de Lowe o eliminant outliers.

\newpage
\section{Homografia}

	L'objectiu principal del programa es trobar una part d'una imatge en una altre imatge diferent i per fer això utilitzarem la homografia. Trobant la relació entre els píxels de les dues imatges podrem
	reprojectar el pla d'una imatge en l'altre i trobar el punt on volem dirigir el robot.\\
	\begin{figure}[H]
		\centering
		\includegraphics[width=0.7\textwidth]{images/homography}
		\caption{Homografia}
	\end{figure}
	\noindent
	A l'hora de buscar la homografia aplicarem RANSAC (Random Sample Consensus), un algorisme que ens permetrà eliminar outliers dels match trobats.\\
	\begin{figure}[H]
		\centering
		\includegraphics[width=0.45\textwidth]{images/ransac}
		\caption{Ransac}
	\end{figure}

%	\chapter{Implementació}
%	\section{Programa principal en python}
	Per facilitar la utilització del codi en possibles adaptacions o millores futures, s'ha decidit implementar una petita biblioteca en Python amb totes les funcions necessaries. El programa principal farà ús
	d'aquesta biblioteca, que permetrà:

	\begin{enumerate}
		\item{Pre-processar les imatges}
		\item{Seleccionar la regió d'interès}
		\item{Obtenir els keypoints}
		\item{Extreure les característiques}
		\item{Fer matching de característiques}
		\item{Homografia: Obtenir la regió/punt demanat}
		\item{Obtenir l'angle de rotació pel robot}
	\end{enumerate}

	\subsection{Pre-processat de les imatges}
		S'han provat diverses tècniques de pre-processat com ara filtres gaussians o canvis en el contrast, pero els resultats obtinguts no han sigut satisfactoris i finalment s'ha optat per deixar
		les imatges tal com són.\\\\
		Simplement es transforma la imatge a escala de grisos per poder treballar amb tots els algorismes de visió i es redimensiona per agilitzar la selecció de keypoints, extracció
		de característiques i matching.\\
		\begin{python}
def prep(image):
	img = cv2.resize(image, (0,0), fx=0.3, fy=0.3)
	return img, cv2.cvtColor(img, cv2.COLOR_BGR2GRAY)
		\end{python}
	\subsection{Selecció de la regió d'interès}
		Per tal de poder seleccionar la regió d'interès amb facilitat, s'ha definit la funció \textbf{selectROI(image)}, que permet a l'usuari seleccionar una regió rectangular de la imatge donada
		com a paràmetre. La imatge resultant serà aquesta selecció.\\
		\begin{python}
def click_and_crop(event, x, y, flags, param):
	global refPt, cropping, sel_rect_endpoint, img
 
	if event == cv2.EVENT_LBUTTONDOWN:	# Initial coordinates. Cropping = true
		cropping = True
		refPt = [(x, y)] 
	elif event == cv2.EVENT_LBUTTONUP:	# End coordinates. Cropping = false (done)
		cropping = False
		refPt.append((x, y)) 
		clone = img.copy()
		cv2.rectangle(clone, refPt[0], refPt[1], (0, 255, 255), 2)	# Draw a rectangle (ROI)
		cv2.imshow("image", clone)
	elif event == cv2.EVENT_MOUSEMOVE and cropping:	# Update position (moving rectangle)
		sel_rect_endpoint = [(x, y)]

def selectROI(image):
	global img, refPt, sel_rect_endpoint
	img = image ###
	cv2.namedWindow("image")
	cv2.setMouseCallback("image", click_and_crop)
	cv2.imshow('image', img)

	while True:
		if not cropping:
			sel_rect_endpoint = []
		elif cropping and sel_rect_endpoint:	# Display rectangle (moving)
			clone = img.copy()
			cv2.rectangle(clone, refPt[0], sel_rect_endpoint[0], (0, 255, 0), 1)
			cv2.imshow('image', clone)
		if (cv2.waitKey(1) & 0xFF) == ord("c"):
			break

	cv2.destroyAllWindows()
	if len(refPt) == 2:
		img = img[refPt[0][1]:refPt[1][1], refPt[0][0]:refPt[1][0]]
	return img
		\end{python}
	\subsection{Obtenció de keypoints}
		Per obtenir els punts d'interès d'una imatge, s'utilitzaràn diversos algorismes de visió per computador.\\\\
		La funció point{\_}selection(gray, alg) serà l'encarregada de cridar l'algoritme desitjat segons el que vulgui l'usuari (per defecte s'utilitzarà Harris).
		Els paràmetres necessaris seràn una imatge en escala de grisos i l'algorisme desitjat. La funció retornarà un array amb els keypoints trobats. \\
		\begin{python}
def point_selection(gray, alg):
	kp = []
	...
	return kp
		\end{python}

		\subsubsection{SIFT}
		Una de les opcions serà utilitzar SIFT (DoG). Per fer això simplement cridarem la funció d'OpenCV detect() en la instància de SIFT creada.\\
		\begin{python}
	if alg == _SIFT:
		sift = cv2.xfeatures2d.SIFT_create()
		kp = sift.detect(gray,None)
		\end{python}

		\subsubsection{ORB}
		Per utilitzar ORB també cridarem la funció detect() d'OpenCV, pero en aquest cas haurem de modificar una mica els paràmetres per defecte de la creació de l'objecte ORB, ja que els resultats obtinguts
		en primer moment no eren gaire bons.\\
		\begin{python}
	elif alg == _ORB:
		orb = cv2.ORB_create(nfeatures = 5000, nlevels = 8, edgeThreshold = 8,
		 patchSize = 8, fastThreshold = 5)
		kp = orb.detect(gray,None)
		\end{python}
\newpage
		\subsubsection{Harris}
		En el cas de Harris, el detector no utilitza diferents escales i simplement detecta ``corners'' en la imatge. Per tant, utilitzarem la funció d'OpenCV pyrDown() per reduir
		la mida de la imatge (1/2) i aplicar Harris en diverses escales.\\\\
		Per detectar els keypoints farem servir la funció goodFeaturesToTrack() en comptes del detector de corners de Harris, ja que permet obtenir els punts més facilment i
		també té la opció d'utilitzar Shi-Tomasi si ens interessés.\\
		\begin{python}
	elif alg == _HARRIS:
		G = gray.copy()
		for i in range(5):
			if i != 0:
				G = cv2.pyrDown(G)
			scale = 2**(i)
			corners = cv2.goodFeaturesToTrack(image=G,maxCorners=1000,
				qualityLevel=0.01,minDistance=scale,useHarrisDetector=1, k=0.04)
			corners = np.int0(corners)
			for corner in corners:
				x,y = corner.ravel()
				k = cv2.KeyPoint(x*scale, y*scale, scale)
				kp.append(k)
		\end{python}

		\subsubsection{MSER}
		Per últim, tindrem la opció d'utilitzar MSER. Tal com fem amb SIFT i ORB, també farem servir la funció detect().\\
		\begin{python}
	elif alg == _MSER:
		mser = cv2.MSER_create()
		kp = mser.detect(gray,None)
		\end{python}
\newpage
	\subsection{Extracció de característiques}
		De la mateixa manera en que podem aconseguir keypoints, OpenCV també disposa de diversos algorismes d'extracció de característiques a partir dels keypoints d'una imatge.\\\\
		La funció creada feature{\_}detection(image, kp, alg) serà l'encarregada d'extreure les característiques utilitzant algun dels algorismes disponibles. Els paràmetres necessaris de la funció són:
		la imatge en escala de grisos, l'array de keypoints obtinguts i l'algorisme desitjat. Retornarà tant els descriptors com els keypoints.\\

		\begin{python}
def feature_extraction(image, kp, alg):
	des = []
	...
	return kp, des
		\end{python}
		\ \\Els algorismes que podrà escollir l'usuari són:
		\begin{itemize}
			\item{SIFT}
			\item{SURF}
			\item{ORB}
			\item{BRISK}
			\item{LATCH}
			\item{DAISY}
		\end{itemize}
	\ \\S'utilitzaran els algorismes ja implementats en la biblioteca OpenCV de la següent manera:\\

		\begin{python}
	elif alg == _LATCH:
		latch = cv2.xfeatures2d.LATCH_create()
		kp, des = latch.compute(image, kp)
		\end{python}

\newpage
	\subsection{Matching i homografia}
La funció retornarà les coordenades de destí (x,y) i una imatge amb els matches i la regió de destí pintats.\\
		\begin{python}
def matching(img1, img2, des1, des2, kp1, kp2, fe):
	draw_params = dict(matchColor = (0,255,0), singlePointColor = None,
					matchesMask = matchesMask, flags = 2)
	return x, y, cv2.drawMatches(img1,kp1,img2,kp2,good,None,**draw_params)
		\end{python}

		\subsubsection{Matching}
		En el cas dels descriptors binaris, utilitzarem la distancia Hamming, mentre que per la resta s'utilitzarà... En els dos casos, s'utilitzarà la funció BFMatcher(), que aplicarà el matching
		per força bruta.\\
		\begin{python}
	if fe == _LATCH or fe == _ORB or fe == _BRISK:
		bf = cv2.BFMatcher(cv2.NORM_HAMMING)
	else:
		bf = cv2.BFMatcher()

	matches = bf.knnMatch(des1, des2, k=2)
		\end{python}

		\subsubsection{Ratio test}
Un cop obtinguts els "matches", aplicarem el ratio test per descartar alguns "matches".\\
		\begin{python}
	good = []
	for m,n in matches:
		if m.distance < 0.75*n.distance:
			good.append(m)
		\end{python}

\newpage
		\subsubsection{Homografia}
Si hi ha prou coincidencies (considerem acceptable com a mínim 10), es buscarà la homografia i es pintarà la regió trobada en la imatge. S'aplicarà Ransac. \\
		\begin{python}
	if len(good) >= MIN_MATCH_COUNT:
		src_pts = np.float32([ kp1[m.queryIdx].pt for m in good ]).reshape(-1,1,2)
		dst_pts = np.float32([ kp2[m.trainIdx].pt for m in good ]).reshape(-1,1,2)
		M, mask = cv2.findHomography(src_pts, dst_pts, cv2.RANSAC, 5.0)
		matchesMask = mask.ravel().tolist()
		h,w,_ = img1.shape
		pts = np.float32([ [0,0],[0,h-1],[w-1,h-1],[w-1,0] ]).reshape(-1,1,2)
		dst = cv2.perspectiveTransform(pts,M)
		img2 = cv2.polylines(img2,[np.int32(dst)],True,255,3, cv2.LINE_AA)
		\end{python}

		\subsubsection{Coordenades de destí}
Per obtenir el punt on volem que es dirigeixi el robot agafarem els valors de dst i retornarem el punt mig.\\
		\begin{python}
		x1, y1 = np.int32(dst)[0].ravel()
		x2, y2 = np.int32(dst)[1].ravel()
		x3, y3 = np.int32(dst)[2].ravel()
		x4, y4 = np.int32(dst)[3].ravel()
		x = (x1+x2+x3+x4)/4
		y = (y1+y2+y3+y4)/4
		\end{python}

	\subsection{Angle de gir}
		Un cop obtingudes les coordenades de la imatge, s'obtindrà l'angle de gir necessari pel robot a partir de l'angle de visió de la càmera i les dimensions de la imatge.\\
		\begin{python}
def getAngle(aV, w, x):
	if x == -1:
		return 0
	else:
		return (aV*x / w) - (aV/2)
		\end{python}

\section{Aplicació web}
	L'aplicació web s'ha desenvolupat amb Flask (Python).
	\subsection{Pujar imatge al servidor}
		Per poder utilitzar la càmera del mòbil s'ha hagut d'afegir el permís necessari.\\\\
		Com que Android ja disposa d'una Activitat que ens permet llançar la càmera desde una altre aplicació, nomès ha sigut necessari fer l'Intent corresponent.\\
		\begin{python}
app.config['UPLOAD_FOLDER'] = 'static/'
app.config['ALLOWED_EXTENSIONS'] = set(['png', 'jpg', 'jpeg', 'gif'])

# For a given file, return whether it's an allowed type or not
def allowed_file(filename):
	return '.' in filename and 
		filename.rsplit('.', 1)[1] in app.config['ALLOWED_EXTENSIONS']

@app.route("/update")
def update():
	return render_template('upload.html')

@app.route('/upload', methods=['POST'])
def upload():
	file = request.files['file']
	if file and allowed_file(file.filename):
		filename = secure_filename(file.filename)
		file.save(os.path.join(app.config['UPLOAD_FOLDER'], filename))
		return redirect(url_for('uploaded_file', filename=filename))

@app.route('/uploads/<filename>')
def uploaded_file(filename):
	return send_from_directory(app.config['UPLOAD_FOLDER'], filename)
		\end{python}

		\begin{txt}
<form action="upload" method="post" 
	enctype="multipart/form-data">
	<div class="col-lg-6 col-sm-6 col-12">
		<label class="input-group-btn">
			<span class="btn btn-primary">
				Browse&hellip; <input type="file" 
					name="file" style="display: none;">
			</span>
		</label>
	</div>
	<div class="col-lg-6 col-sm-6 col-12">
		<input class="btn btn-primary" type="submit"
			value="Pujar imatge">
	</div>
</form>
		\end{txt}

	\subsection{Selecció de la regió d'interès}
		Per poder utilitzar la càmera del mòbil s'ha hagut d'afegir el permís necessari.\\\\
		Com que Android ja disposa d'una Activitat que ens permet llançar la càmera desde una altre aplicació, nomès ha sigut necessari fer l'Intent corresponent.\\
		\begin{txt}
<form action="send" method="post" >
	<input type="hidden" name="x1" value="" />
	<input type="hidden" name="y1" value="" />
	<input type="hidden" name="x2" value="" />
	<input type="hidden" name="y2" value="" />
	<input type="hidden" name="width" value="" />
	<input type="hidden" name="height" value="" />
	<input class="btn btn-primary" type="submit"
		value="Send" />
</form>

<script type="text/javascript">
	jQuery(document).ready(function () {
		jQuery('img#scene').imgAreaSelect({
			handles: true,
			persistent: true,
			imageHeight: 2448,
			imageWidth: 3264,
			x1: 100, y1: 100, x2: 300, y2: 300,
			onInit: function ( image, selected) {
				jQuery('input[name=x1]').val(selected.x1);
				jQuery('input[name=y1]').val(selected.y1);
				jQuery('input[name=x2]').val(selected.x2);
				jQuery('input[name=y2]').val(selected.y2);
				jQuery('input[name=width]')
					.val(selected.width);
				jQuery('input[name=height]')
					.val(selected.height);
			},
			onSelectEnd: function ( image, selected) {
				jQuery('input[name=x1]').val(selected.x1);
				jQuery('input[name=y1]').val(selected.y1);
				jQuery('input[name=x2]').val(selected.x2);
				jQuery('input[name=y2]').val(selected.y2);
				jQuery('input[name=width]')
					.val(selected.width);
				jQuery('input[name=height]')
					.val(selected.height);
			}
		});
	});
 </script>
		\end{txt}

		\begin{python}
@app.route("/")
def index():
	return render_template('index.html')

@app.route('/send', methods=['POST'])
def send():
	x1 = int(request.form['x1'])
	x2 = int(request.form['x2'])
	y1 = int(request.form['y1'])
	y2 = int(request.form['y2'])

	img = cv2.imread(imgPath)
	img = img[x1:x2, y1:y2]
	imgRobot = cv2.imread(imgRobotPath)
	imgRes, x, y = vc.getResult(img, imgRobot, vc._SIFT)
	cv2.imwrite("static/result.png", imgRes)

	return render_template('result.html', x=x, y=y)
		\end{python}


\section{Aplicació mòbil}
	L'aplicació mòbil s'ha desenvolupat amb Android Studio i el llenguatge de programació Java (amb l'SDK d'Android).
	\subsection{Capturar imatge}
		Per poder utilitzar la càmera del mòbil s'ha hagut d'afegir el permís necessari.\\\\
		Com que Android ja disposa d'una Activitat que ens permet llançar la càmera desde una altre aplicació, nomès ha sigut necessari fer l'Intent corresponent.\\
		\begin{java}
class MyClass(Yourclass):
  def __init__(self, my, yours):
    String = "String"
    String2 = '5 1 2 3 4'
    print String2
import numpy as np #Comment1
  # Comment2
		\end{java}
	\subsection{Selecció d'imatges de la galeria}
		Per poder utilitzar la càmera del mòbil s'ha hagut d'afegir el permís necessari.\\\\
		Com que Android ja disposa d'una Activitat que ens permet llançar la càmera desde una altre aplicació, nomès ha sigut necessari fer l'Intent corresponent.\\
		\begin{java}
class MyClass(Yourclass):
  def __init__(self, my, yours):
    String = "String"
    String2 = '5 1 2 3 4'
    print String2
import numpy as np #Comment1
  # Comment2
		\end{java}
	\subsection{Enviament de dades al servidor}
		Per poder enviar dades al servidor, serà necessari l'ús d'una connexió a Internet. Per tant, s'haurà d'habilitar el permís necessari.\\\\
		Com que Android ja disposa d'una Activitat que ens permet llançar la càmera desde una altre aplicació, nomès ha sigut necessari fer l'Intent corresponent.\\
		\begin{java}
class MyClass(Yourclass):
  def __init__(self, my, yours):
    String = "String"
    String2 = '5 1 2 3 4'
    print String2
import numpy as np #Comment1
  # Comment2
		\end{java}

%	\chapter{Resultats obtinguts}
%	\newcolumntype{x}[1]{>{\centering\arraybackslash\hspace{0pt}}p{#1}}
\newcolumntype{M}[1]{>{\centering\arraybackslash\hspace{0pt}}m{#1}}
%\definecolor{myBlue}{RGB}{210, 230, 255}
\definecolor{myBlue}{RGB}{217, 230, 242}

En aquest capítol es detallen els experiments realitzats i els resultats obtinguts a partir d'aquests.

\section{Experiments realitzats}
	Podeu trobar el codi dels scripts realitzats a l'anex del treball.
	\subsection{Comparació detectors de keypoints}
		Tot i que la velocitat d'execució no és un factor determinant pel projecte, s'ha realitzat una comparació inicial de la velocitat d'execució dels algorismes de detecció de keypoints.\\\\
		Es comparen els següents algorismes:
		\begin{itemize}
			\item{Harris}
			\item{SIFT}
			\item{SURF}
			\item{ORB}
			\item{MSER}
		\end{itemize}
		Per mesurar el temps d'execució, s'agafa la mitja de 5 execucions de la funció d'obtenció de keypoints utilitzada.
	\subsection{Comparació detecció i extracció de keypoints}
		En aquest experiment s'analitzen els diversos algorismes de detecció i extracció de keypoints, tant en la velocitat d'execució com en fiabilitat, que serà el més important pel sistema desenvolupat.\\\\
		Es compararan els següents algorismes:
		\begin{itemize}
			\item{Harris + SIFT}
			\item{Harris + LATCH}
			\item{SIFT + SIFT}
			\item{SIFT + LATCH}
			\item{ORB + ORB}
			\item{ORB + BRISK}
			\item{MSER + SURF}
			\item{MSER + DAISY}
		\end{itemize}
\newpage
\section{Resultats i comparació d'algorismes}
	\subsection{Detectors de keypoints}
	En primer lloc s'han realitzat proves amb imatges típiques de visió per computador.
		\begin{figure}[!htb]
			\minipage{0.32\textwidth}
				\includegraphics[width=\linewidth]{images/RobotKp}
				\label{fig:awesome_image1}
			\endminipage\hfill
			\minipage{0.32\textwidth}
				\includegraphics[width=\linewidth]{images/RobotKp}
				\label{fig:awesome_image2}
			\endminipage\hfill
			\minipage{0.32\textwidth}%
				\includegraphics[width=\linewidth]{images/RobotKp}
				\label{fig:awesome_image3}
			\endminipage
			\caption{Pati}
		\end{figure}

		\begin{table}[H]
			\begin{center}
				\rowcolors{3}{}{myBlue}
				%\begin{tabular}{l | !{\vrule width -1pt}c !{\vrule width -1pt}c | !{\vrule width -1pt}c !{\vrule width -1pt}c | !{\vrule width -1pt}c !{\vrule width -1pt}c}
				\begin{tabular}{l | c c | c c | c c}
					& \multicolumn{2}{c|}{\textbf{Motos}} & \multicolumn{2}{c|}{\textbf{Barco}} & \multicolumn{2}{c}{\textbf{Ubc}} \\
					\textbf{Algorismes} & \textbf{Punts} & \textbf{Temps} & \textbf{Punts} & \textbf{Temps} & \textbf{Punts} & \textbf{Temps} \\ \hline
					Harris & 0 & 0ms & 0 & 0ms & 0 & 0ms \\
					SIFT & 0 & 0ms & 0 & 0ms & 0 & 0ms \\
					SURF & 0 & 0ms & 0 & 0ms & 0 & 0ms \\
					ORB & 0 & 0ms & 0 & 0ms & 0 & 0ms \\
					MSER & 0 & 0ms & 0 & 0ms & 0 & 0ms \\
				\end{tabular}
			\end{center}
			\caption{Detectors de keypoints - comparació}
		\end{table}
		\noindent
		També s'ha provat el sistema amb imatges reals d'entorns coneguts (la universitat i casa meva).

		\begin{figure}[!htb]
			\minipage{0.32\textwidth}
				\includegraphics[width=\linewidth]{images/RobotKp}
				\label{fig:awesome_image1}
			\endminipage\hfill
			\minipage{0.32\textwidth}
				\includegraphics[width=\linewidth]{images/RobotKp}
				\label{fig:awesome_image2}
			\endminipage\hfill
			\minipage{0.32\textwidth}%
				\includegraphics[width=\linewidth]{images/RobotKp}
				\label{fig:awesome_image3}
			\endminipage
			\caption{Pati}
		\end{figure}

		\begin{table}[H]
			\begin{center}
				\rowcolors{3}{}{myBlue}
				%\begin{tabular}{l | !{\vrule width -1pt}c !{\vrule width -1pt}c | !{\vrule width -1pt}c !{\vrule width -1pt}c | !{\vrule width -1pt}c !{\vrule width -1pt}c}
				\begin{tabular}{l | c c | c c | c c}
					& \multicolumn{2}{c|}{\textbf{Universitat}} & \multicolumn{2}{c|}{\textbf{Menjador}} & \multicolumn{2}{c}{\textbf{Pati}} \\
					\textbf{Algorismes} & \textbf{Punts} & \textbf{Temps} & \textbf{Punts} & \textbf{Temps} & \textbf{Punts} & \textbf{Temps} \\ \hline
					Harris & 0 & 0ms & 0 & 0ms & 0 & 0ms \\
					SIFT & 0 & 0ms & 0 & 0ms & 0 & 0ms \\
					SURF & 0 & 0ms & 0 & 0ms & 0 & 0ms \\
					ORB & 0 & 0ms & 0 & 0ms & 0 & 0ms \\
					MSER & 0 & 0ms & 0 & 0ms & 0 & 0ms \\
				\end{tabular}
			\end{center}
			\caption{Detectors de keypoints - comparació}
		\end{table}
\newpage
	\subsection{Detecció i extracció de keypoints}
		Tal com s'ha fet en l'experiment anterior, primer s'ha provat el sistema amb les imatges de visió.

		\begin{figure}[!htb]
			\minipage{0.24\textwidth}
				\includegraphics[width=\linewidth]{images/RobotKp}
				\label{fig:awesome_image1}
			\endminipage\hfill
			\minipage{0.24\textwidth}
				\includegraphics[width=\linewidth]{images/RobotKp}
				\label{fig:awesome_image2}
			\endminipage\hfill
			\minipage{0.24\textwidth}
				\includegraphics[width=\linewidth]{images/RobotKp}
				\label{fig:awesome_image3}
			\endminipage\hfill
			\minipage{0.24\textwidth}
				\includegraphics[width=\linewidth]{images/RobotKp}
				\label{fig:awesome_image3}
			\endminipage
			\caption{Pati}
		\end{figure}

		\begin{table}[H]
			\begin{center}
				\rowcolors{3}{}{myBlue}
				\begin{tabular}{l | c c c | c c c}
					& \multicolumn{3}{c|}{\textbf{Cotxes}} & \multicolumn{3}{c}{\textbf{Graffiti}} \\
					\textbf{Algorismes} & \textbf{Matches} & \textbf{\%} & \textbf{Temps} & \textbf{Matches} & \textbf{\%} & \textbf{Temps} \\ \hline
					Harris + SIFT & 0 & 0\% & 0ms & 0 & 0\% & 0ms \\
					Harris + LATCH & 0 & 0\% & 0ms & 0 & 0\% & 0ms \\
					SIFT + SIFT & 0 & 0\% & 0ms & 0 & 0\% & 0ms \\
					SIFT + LATCH & 0 & 0\% & 0ms & 0 & 0\% & 0ms \\
					ORB + ORB & 0 & 0\% & 0ms & 0 & 0\% & 0ms \\
					ORB + BRISK & 0 & 0\% & 0ms & 0 & 0\% & 0ms \\
					MSER + SURF & 0 & 0\% & 0ms & 0 & 0\% & 0ms \\
					MSER + DAISY & 0 & 0\% & 0ms & 0 & 0\% & 0ms \\
				\end{tabular}
			\end{center}
			\caption{Matching - comparació}
		\end{table}
		\noindent
		Les proves amb imatges reals han utilitzat les següents fotografies:

		\begin{figure}[!htb]
			\minipage{0.24\textwidth}
				\includegraphics[width=\linewidth]{images/RobotKp}
				\label{fig:awesome_image1}
			\endminipage\hfill
			\minipage{0.24\textwidth}
				\includegraphics[width=\linewidth]{images/RobotKp}
				\label{fig:awesome_image2}
			\endminipage\hfill
			\minipage{0.24\textwidth}
				\includegraphics[width=\linewidth]{images/RobotKp}
				\label{fig:awesome_image3}
			\endminipage\hfill
			\minipage{0.24\textwidth}
				\includegraphics[width=\linewidth]{images/RobotKp}
				\label{fig:awesome_image3}
			\endminipage
			\caption{Pati}
		\end{figure}

		\begin{table}[H]
			\begin{center}
				\rowcolors{3}{}{myBlue}
				\begin{tabular}{l | c c c | c c c}
					& \multicolumn{3}{c|}{\textbf{Universitat}} & \multicolumn{3}{c}{\textbf{Menjador}} \\
					\textbf{Algorismes} & \textbf{Matches} & \textbf{\%} & \textbf{Temps} & \textbf{Matches} & \textbf{\%} & \textbf{Temps} \\ \hline
					Harris + SIFT & 0 & 0\% & 0ms & 0 & 0\% & 0ms \\
					Harris + LATCH & 0 & 0\% & 0ms & 0 & 0\% & 0ms \\
					SIFT + SIFT & 0 & 0\% & 0ms & 0 & 0\% & 0ms \\
					SIFT + LATCH & 0 & 0\% & 0ms & 0 & 0\% & 0ms \\
					ORB + ORB & 0 & 0\% & 0ms & 0 & 0\% & 0ms \\
					ORB + BRISK & 0 & 0\% & 0ms & 0 & 0\% & 0ms \\
					MSER + SURF & 0 & 0\% & 0ms & 0 & 0\% & 0ms \\
					MSER + DAISY & 0 & 0\% & 0ms & 0 & 0\% & 0ms \\
				\end{tabular}
			\end{center}
			\caption{Matching - comparació}
		\end{table}

	\chapter{Gestió econòmica}
	\def\arraystretch{1.4}
\definecolor{tableHeader}{RGB}{53, 104, 151}
\definecolor{total2}{RGB}{150, 200, 230}
\definecolor{total}{RGB}{240, 240, 240} %GRIS
\definecolor{myGray}{RGB}{236, 242, 248}
\newcolumntype{x}[1]{>{\centering\arraybackslash\hspace{0pt}}m{#1}}

\label{sec:Costos}

%A continuació es detallen els costos necessaris per la realització d'aquest treball.
\section{Recursos de programari}
	Tot el \textit{software} utilitzat en aquest projecte és gratuït i de codi obert. Per tant, el programari no suposarà cap despesa. Podeu trobar el llistat del programari utilitzat
	a la taula \ref{table:programari} (Recursos de programari).

\section{Recursos humans}
	El projecte el desenvoluparà una sola persona, que assumirà diversos rols durant la realització d'aquest. Tenint en compte les tasques descrites a la secció \ref{sec:planificacio},
	les hores de treball queden repartides de la següent manera:
	\begin{table}[H]
		\begin{center}
			%\rowcolors{2}{myGray}{white}
			\begin{tabular}{m{4.2cm} !{\vrule width -1pt}x{2.4cm} !{\vrule width -1pt}x{2.4cm} !{\vrule width -1pt}x{3cm}}
				%\rowcolor{tableHeader}
				\textbf{Tasca} & \textbf{Cap} & \textbf{Analista} & \textbf{Programador} \\ \hline
				Preparació de l'entorn & 3h & & 2h \\
				Curs de GEP & 75h & & \\
				Implementació i proves & & 40h & 200h \\
				Conclusions i resultats & & 30h & \\
				Ampliacions (opcional) & & & 30h\\
				Redacció memòria & 40h & & \\
				Preparació defensa & 30h & & \\
				\noalign{\vskip 4mm}
				\rowcolor{total}
				Total & 148h & 70h & 232h
			\end{tabular}
		\end{center}
		\caption{Recursos humans (hores)}
	\end{table}
	\noindent{Suposem uns costos de 25€/h pel cap de projecte, 20€/h per l'analista i 15€/h pel programador/\textit{tester}.}
	\begin{table}[H]
		\begin{center}
			%\rowcolors{2}{myGray}{white}
			\begin{tabular}{l !{\vrule width -1pt}r !{\vrule width -1pt}r !{\vrule width -1pt}r}
				%\rowcolor{tableHeader}
				\textbf{Rol} & \textbf{Hores} & \textbf{Cost/hora} & \textbf{Cost total} \\ \hline
				Cap de projecte & 148h & 25€/h & 3700€ \\
				Analista & 70h & 20€/h & 1400€ \\
				Programador & 230h & 15€/h & 3450€ \\
				%\textit{Tester} & h & 15€/h & € \\
				\noalign{\vskip 4mm}
				\rowcolor{total}
				Total & & & 8550€
			\end{tabular}
		\end{center}
		\caption{Recursos humans (costos)}
	\end{table}


\section{Recursos de maquinari}
	El \textit{hardware} necessari per a la realització del treball serà només un ordinador (usat durant tot el projecte) i una càmera (per la fase d'implementació/proves).
	\begin{table}[H]
		\begin{center}
			%\rowcolors{2}{myGray}{white}
			\begin{tabular}{l !{\vrule width -1pt}r !{\vrule width -1pt}r !{\vrule width -1pt}r !{\vrule width -1pt}r}
				%\rowcolor{tableHeader}
				\textbf{Producte} & \textbf{Preu} & \textbf{Ús} & \textbf{Vida útil} & \textbf{Amortització} \\ \hline
				Ordinador personal & 500€ & 5 mesos & 5 anys & 41,70€ \\
				Smartphone & 39€ & 1 mes & 3 anys & 1,08€ \\
				\noalign{\vskip 4mm}
				\rowcolor{total}
				Total &  &  &  & 42,78€ \\
			\end{tabular}
		\end{center}
		\caption{Recursos de maquinari (costos)}
	\end{table}

\section{Costos indirectes}
	També es tindran en compte els costos indirectes més importants: la connexió a Internet i el consum elèctric. La connexió a Internet costarà 40€ al mes (considerem 240 hores) i
	l'electricitat 0,141033€/kWh (considerem la potència 0,2kW).
	\begin{table}[H]
		\begin{center}
			%\rowcolors{2}{myGray}{white}
			\begin{tabular}{l !{\vrule width -1pt}r !{\vrule width -1pt}r !{\vrule width -1pt}r}
				%\rowcolor{tableHeader}
				\textbf{Tipus} & \textbf{Temps} & \textbf{Cost} & \textbf{Cost total} \\ \hline
				Electricitat & 450h & 0,028€/h & 12,69€ \\
				Accès a Internet & 450h & 0,17€/h & 75€ \\
				\noalign{\vskip 4mm}
				\rowcolor{total}
				Total & & & 87,69€
			\end{tabular}
		\end{center}
		\caption{Costos indirectes}
	\end{table}

\section{Imprevistos}
	Es podria donar el cas que el projecte ocupi més temps de l'esperat, pel que es considerarà un extra de 30 hores de treball, que es dividirien entre el programador i el \textit{tester}.
	Això suposaria un increment de 600€ en el pressupost. \\\\
	No es tindran en compte possibles fallades de maquinari, ja que l'ordinador principal amb què es treballa està en garantia i també es disposa d'altres ordinadors.

\section{Contingència}
	Com a mesura de contingència, s'estableix un marge del 5\%.

\section{Costos totals}
	\begin{table}[H]
		\begin{center}
			%\rowcolors{2}{myGray}{white}
			\begin{tabular}{p{8cm}  !{\vrule width -1pt}r}
				%\rowcolor{tableHeader}
				\textbf{Tipus} & \textbf{Cost estimat} \\ \hline
				Recursos humans & 8550€ \\
				Recursos de programari & 0€ \\
				Recursos de maquinari & 42,78€ \\
				Costos indirectes & 87,69€ \\
				Imprevistos & 600€ \\
				Contingència (5\%) & 464,02€ \\
				\noalign{\vskip 4mm}
				\rowcolor{total}
				Total & 9744,5€
			\end{tabular}
		\end{center}
		\caption{Costos totals}
	\end{table}

\section{Control de gestió}
	Després de cada tasca es farà una valoració del pressupost i es revisarà si és necessari.
	També es durà a terme un control de les hores de treball per cada rol mitjançant un full de càlcul, que s'anirà actualitzant cada dia de treball.\\\\
	Es calcularà la desviació en mà d'obra, programari, maquinari i altres costos (cost estimat - cost real).

	\chapter{Informe de sostenibilitat}
	[MATRIU]

\section{Sostenibilitat econòmica}
	Lorem ipsum dolor sit amet, consectetur adipiscing elit. Pellentesque quis pellentesque odio. Sed hendrerit nisi ut metus malesuada placerat. Aenean a turpis tempus, auctor elit nec, elementum ante. Pellentesque faucibus malesuada pulvinar. Duis tristique est nibh, in dapibus est iaculis ut. Vestibulum gravida congue mi, a sodales est imperdiet at. Aliquam varius condimentum nisi, et pulvinar orci egestas eget. Sed pharetra metus nec enim fermentum commodo. Aenean auctor purus sit amet felis imperdiet semper. Ut scelerisque, ex eu tristique pretium, tellus purus vehicula magna, sed ornare dolor sem ut sapien. Morbi sit amet odio odio. Maecenas sodales nulla nec velit tempor, eu aliquam sapien cursus. Sed volutpat libero a nunc luctus, sagittis commodo ipsum ultrices.

\section{Sostenibilitat social}
	Phasellus consectetur tincidunt nunc, quis ultrices sapien rutrum non. Sed posuere lacinia mattis. Nunc fringilla ultricies dolor a ullamcorper. Praesent eu odio et dui aliquam aliquam eget a nibh. Nullam metus eros, convallis nec rhoncus eu, fermentum nec libero. Mauris finibus nunc in dapibus rhoncus. Cras ac lacus at nunc elementum tristique nec sed arcu. Aliquam ex neque, finibus sed laoreet ut, fringilla at nulla. 

\section{Sostenibilitat ambiental}
	Sed mattis turpis a nisi sodales, nec sodales nisi finibus. Mauris et nisi sit amet velit gravida facilisis et in lectus. Suspendisse sodales erat non semper fermentum. Mauris nec lacus non sapien fermentum posuere. Sed hendrerit nisl ac lectus consectetur sagittis. Cras vehicula lacus eget augue feugiat tincidunt. Vivamus vulputate enim velit, eget volutpat dolor dignissim at. 

%	\chapter{Conclusions}
%	En aquest apartat es descriuen les conclusions extretes despres de realitzar aquest treball final de grau. En primer lloc es descriuràn les conclusions tècniques, seguides de les conclusions personals
i finalment es detallarà els plans de treball futurs i les possibles ampliacions i millores del sistema.\\\\
En general, l'objectiu principal del projecte, crear un sistema d'autolocalització, s'ha complert.

\section{Conclusions tècniques}
	Veient els resultats dels experiments realitzats, considero que la taxa d'encert del sistema és acceptable. Tot i així, crec que encara hi ha molt marge de millora.\\\\
	Analitzant els resultats, considero que Harris és el detector de punts d'interés que obté punts més robustos i juntament amb el descriptor de SIFT aconsegueix bons aparellaments. Per altra banda,
	ORB sembla una bona alternativa en casos en que el temps d'execució sigui crític.
\section{Conclusions personals}
	Aquest treball m'ha permes profunditzar els meus coneixements sobre visió per computador. També m'ha ajudat en poder planificar projectes, gestionant recursos i establint objectius.
	Considero que de cara al món laboral la realització d'aquest projecte ha estat una experiencia molt positiva.

\newpage
\section{Treball futur}
	En principi no està previst continuar amb el treball en un futur, pero el codi serà públic i no es descarta continuar amb el projecte més endavant (personalment o a través d'altres persones).\\\\
	En tot cas, hi ha una serie de possibles millores i ampliacions que caldria mencionar:\\

	\begin{itemize}
		\item{Comparació i anàlisi d'algorismes: És necessari fer un anàlisi més exhaustiu dels diversos algorismes de detecció i extracció de característiques.}
		\item{Diferents imatges: El sistema s'hauria de provar amb imatges diverses. Imatges d'entorns diferents, condicions diferents i capturades amb diverses càmeres.}
		\item{Pre-processat: S'hauria d'investigar amb més profunditat quines tècniques de pre-processat podrien ajudar als algorismes a detectar millors keypoints i característiques.}
		\item{Aplicació mòbil: S'hauria d'acabar l'aplicació d'Android.}
		\item{Provar el sistema en un entorn real i un robot: Com que no es disposava del temps necessari per fer les proves amb un robot, no s'ha pogut experimentar amb el sistema en casos reals. Per tant,
		considero necessaria l'execució de proves amb robots.}
	\end{itemize}


	\printbibliography[heading=bibintoc]
	\cleardoublepage\phantomsection\addcontentsline{toc}{chapter}{\listtablename}
	\listoftables
	\cleardoublepage\phantomsection\addcontentsline{toc}{chapter}{\listfigurename}
	\listoffigures


\end{document}
