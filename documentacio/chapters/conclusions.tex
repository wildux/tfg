En aquest apartat es descriuen les conclusions extretes després de realitzar aquest treball final de grau. En primer lloc es descriuran les conclusions tècniques, seguides de les conclusions personals
i finalment es detallarà els plans de treball futurs i les possibles ampliacions i millores del sistema.\\\\
En general, l'objectiu principal del projecte, crear un sistema d'autolocalització, s'ha complert.

\section{Conclusions tècniques}
	Veient els resultats dels experiments realitzats, considero que la taxa d'encert del sistema és acceptable. Tot i així, crec que encara hi ha molt marge de millora.\\\\
	Analitzant els resultats, considero que Harris és el detector de punts d'interès que obté punts més robustos i juntament amb el descriptor de SIFT aconsegueix bons aparellaments. Per altra banda,
	ORB sembla una bona alternativa en casos en què el temps d'execució sigui crític.
\section{Conclusions personals}
	Aquest treball m'ha permès profunditzar els meus coneixements sobre visió per computador. També m'ha ajudat en poder planificar projectes, gestionant recursos i establint objectius.
	Considero que de cara al món laboral la realització d'aquest projecte ha estat una experiència molt positiva.

\newpage
\section{Treball futur}
	En principi no està previst continuar amb el treball en un futur, però el codi serà públic i no es descarta continuar amb el projecte més endavant (personalment o a través d'altres persones).\\\\
	En tot cas, hi ha una sèrie de possibles millores i ampliacions que caldria mencionar:\\

	\begin{itemize}
		\item{\textbf{Comparació i anàlisi d'algorismes:} És necessari fer una anàlisi més exhaustiva dels diversos algorismes de detecció i extracció de característiques.}
		\item{\textbf{Diferents imatges:} El sistema s'hauria de provar amb imatges diverses. Imatges d'entorns diferents, condicions diferents i capturades amb diverses càmeres.}
		\item{\textbf{Pre-processat:} S'hauria d'investigar amb més profunditat quines tècniques de pre-processat podrien ajudar als algorismes a detectar millors \textit{keypoints} i característiques.}
		\item{\textbf{Aplicació mòbil:} S'hauria d'acabar l'aplicació d'Android.}
		\item{\textbf{Provar el sistema en un entorn real i un robot:} Com que no es disposava del temps necessari per fer les proves amb un robot, no s'ha pogut experimentar amb el sistema en casos reals. Per tant,
		considero necessària l'execució de proves amb robots.}
	\end{itemize}
