En aquest apartat es descriuen les conclusions extretes despres de realitzar aquest treball final de grau. En primer lloc es descriuràn les conclusions tècniques, seguides de les conclusions personals
i finalment es detallarà els plans de treball futurs i les possibles ampliacions i millores del sistema.\\\\
L'objectiu principal del projecte, crear un sistema d'autolocalització, s'ha complert.

\section{Conclusions tècniques}
	Com es pot veure en els experiments realitzats, la taxa d'encert dels match obtinguts amb els diversos algorismes de detecció i extracció de característiques és molt baixa.
\section{Conclusions personals}
	Aquest treball m'ha permes profunditzar els meus coneixements sobre visió per computador.

\newpage
\section{Treball futur}
	En principi no està previst continuar amb el treball en un futur, pero el codi serà públic i no es descarta continuar amb el projecte més endavant (personalment o a través d'altres persones).\\\\
	En tot cas, hi ha una serie de possibles millores i ampliacions que caldria mencionar:\\
	
	\begin{itemize}
		\item{Provar el sistema en un entorn real i un robot: Com que no es disposava del temps necessari per fer les proves amb un robot, no s'ha pogut experimentar amb el sistema en casos reals. Per tant,
		considero necessaria l'execució de proves amb robots.}
		\item{Comparació i anàlisi d'algorismes: És necessari fer un anàlisi més exhaustiu dels diversos algorismes de detecció i extracció de característiques.}
		\item{Pre-processat: S'hauria d'investigar amb més profunditat quines tècniques de pre-processat podrien ajudar als algorismes a detectar millors keypoints i característiques.}
		\item{Servidor: El codi s'hauria de posar en un servidor amb el que l'aplicatiu mòbil es connectaria.}
	\end{itemize}
