\newcolumntype{x}[1]{>{\centering\arraybackslash\hspace{0pt}}p{#1}}
\newcolumntype{M}[1]{>{\centering\arraybackslash\hspace{0pt}}m{#1}}
%\definecolor{myBlue}{RGB}{210, 230, 255}
\definecolor{myBlue}{RGB}{217, 230, 242}

\label{sec:Resultats}

En aquest capítol es detallen els experiments realitzats i els resultats obtinguts a partir d'aquests.

\section{Experiments realitzats}
	S'han realitzat diversos experiments per comprovar tant la detecció de \textit{keypoints} com la fiabilitat del \textit{matching}. Els paràmetres utilitzats en tots els algorismes
	són els descrits al Capítol 6: Implementació.\\\\
	Podeu trobar el codi dels scripts utilitzats per la realització dels experiments al \href{https://github.com/wildux/tfg}{GitHub del projecte}.
	\subsection{Comparació detectors de \textit{keypoints}}
		Tot i que la velocitat d'execució no és un factor determinant pel projecte, s'ha realitzat una comparació inicial de la velocitat d'execució dels algorismes de detecció de \textit{keypoints}.\\\\
		Els algorismes a comparar són:
		\begin{multicols}{3}  
			\begin{itemize}
				\item{Harris}
				\item{SIFT}
				\item{SURF}
				\item{ORB}
				\item{MSER}
			\end{itemize}
		\end{multicols}
		\noindent
		\\{}
		Per mesurar el temps d'execució, s'ha fet la mitjana de 5 execucions de la funció d'obtenció de \textit{keypoints} utilitzada.\\\\
		Un cop obtinguts els temps d'execució, s'ha analitzat la repetibilitat dels detectors a ull, utilitzant imatges similars i sub-imatges.
\newpage
	\subsection{Comparació detecció i extracció de \textit{keypoints}}
		En aquest experiment s'analitzen els diversos algorismes de detecció i extracció de \textit{keypoints}, tant en la velocitat d'execució com en fiabilitat dels aparellaments, que serà el més important
		pel sistema desenvolupat.\\\\
		Es compararan els següents algorismes:
		\begin{multicols}{2} 
			\begin{itemize}
				\item{Harris + SIFT}
				\item{SIFT + SIFT}
				\item{ORB + ORB}
				\item{ORB + BRISK}
			\end{itemize}
		\end{multicols}
		\noindent
		\\{}
		En primer lloc es provaran els algorismes utilitzant sub-imatges. Després es provarà el sistema amb imatges amb poques variacions (canvis d'il·luminació, \textit{blur}...). L'última prova serà amb fotografies
		del campus i els jardins de palau reial, que presenten canvis de perspectiva i zoom entre d'altres.\\\\
		Per comprovar els aparellaments, s'hauran de mirar manualment un per un, ja que les imatges no ens permeten utilitzar cap mecanisme per automatitzar el procés.
\newpage
\section{Resultats i comparació d'algorismes}

	\subsection{Detectors de \textit{keypoints}}
	En primer lloc s'han realitzat proves amb imatges típiques de visió per computador.
		\begin{figure}[!htb]
			\minipage{0.32\textwidth}
				\includegraphics[width=\linewidth]{images/experiments/graf3}
				\label{fig:awesome_image1}
			\endminipage\hfill
			\minipage{0.32\textwidth}
				\includegraphics[width=\linewidth]{images/experiments/boat}
				\label{fig:awesome_image2}
			\endminipage\hfill
			\minipage{0.32\textwidth}%
				\includegraphics[width=\linewidth]{images/experiments/ubc}
				\label{fig:awesome_image3}
			\endminipage
			\caption{Imatges graff, boat i ubc}
		\end{figure}

		\begin{table}[H]
			\begin{center}
				\rowcolors{3}{}{myBlue}
				%\begin{tabular}{l | !{\vrule width -1pt}c !{\vrule width -1pt}c | !{\vrule width -1pt}c !{\vrule width -1pt}c | !{\vrule width -1pt}c !{\vrule width -1pt}c}
				\begin{tabular}{l | c c | c c | c c}
					& \multicolumn{2}{c|}{\textbf{Graff}} & \multicolumn{2}{c|}{\textbf{Boat}} & \multicolumn{2}{c}{\textbf{Ubc}} \\
					\textbf{Algorismes} & \textbf{Punts} & \textbf{Temps (s)} & \textbf{Punts} & \textbf{Temps (s)} & \textbf{Punts} & \textbf{Temps (s)} \\ \hline
					Harris & 587 & 0.0128 & 686 & 0.0071 & 535 & 0.0060 \\
					SIFT & 1096 & 0.0675 & 1599 & 0.0755 & 1284 & 0.0639 \\
					SURF & 1467 & 0.0182 & 1768 & 0.0213 & 1533 & 0.0181 \\
					ORB & 2500 & 0.0102 & 2500 & 0.0128 & 2500 & 0.0116 \\
					MSER & 1044 & 0.2323 & 924 & 0.1586 & 286 & 0.0565 \\
				\end{tabular}
			\end{center}
			\caption{Detectors de \textit{keypoints} - comparació}
		\end{table}
		\noindent
		Amb aquestes imatges, podem veure com Harris, ORB i SURF són força més ràpids que el detector de \textit{keypoints} de SIFT (DoG). MSER sembla ser el més lent en general.\\\\
		S'ha de tenir en compte que el detector ORB s'ha limitat a 2500 punts.\\
\newpage
		\noindent
		També s'ha provat el sistema amb imatges reals d'entorns coneguts (campus nord i jardins de palau reial).

		\begin{figure}[!htb]
			\minipage{0.32\textwidth}
				\includegraphics[width=\linewidth]{images/experiments/uni}
				\label{fig:awesome_image1}
			\endminipage\hfill
			\minipage{0.32\textwidth}
				\includegraphics[width=\linewidth]{images/experiments/jardi2}
				\label{fig:awesome_image2}
			\endminipage\hfill
			\minipage{0.32\textwidth}%
				\includegraphics[width=\linewidth]{images/experiments/uni4}
				\label{fig:awesome_image3}
			\endminipage
			\caption{Imatges UPC i jardins}
		\end{figure}

		\begin{table}[H]
			\begin{center}
				\rowcolors{3}{}{myBlue}
				%\begin{tabular}{l | !{\vrule width -1pt}c !{\vrule width -1pt}c | !{\vrule width -1pt}c !{\vrule width -1pt}c | !{\vrule width -1pt}c !{\vrule width -1pt}c}
				\begin{tabular}{l | c c | c c | c c}
					& \multicolumn{2}{c|}{\textbf{Campus}} & \multicolumn{2}{c|}{\textbf{Jardins}} & \multicolumn{2}{c}{\textbf{Campus 2}} \\
					\textbf{Algorismes} & \textbf{Punts} & \textbf{Temps (s)} & \textbf{Punts} & \textbf{Temps (s)} & \textbf{Punts} & \textbf{Temps (s)} \\ \hline
					Harris & 1553 & 0.0789 & 1282 & 0.0779 & 1302 & 0.0785 \\
					SIFT & 4621 & 0.7358 & 9690 & 0.8172 & 4750 & 0.7405 \\
					SURF & 7676 & 0.1761 & 14888 & 0.2332 & 10720 & 0.1990 \\
					ORB & 2500 & 0.0515 & 2500 & 0.0978 & 2500 & 0.0618 \\
					MSER & 2135 & 0.6790 & 1141 & 0.3930 & 1798 & 0.6397 \\
				\end{tabular}
			\end{center}
			\caption{Detectors de \textit{keypoints} - comparació 2}
		\end{table}
		\noindent
		SURF és el que més \textit{keypoints} detecta, seguit de SIFT. Pel que fa al temps d'execució, SIFT és el més lent, seguit de MSER. El més ràpid amb diferència és ORB, però la
		limitació a 2500 punts podria afectar els resultats. MSER i Harris són els que obtenen menys \textit{keypoints}.

\newpage
%\subsection{Repetibilitat dels detectors de keypoints}
%		\begin{figure}[!htb]
%			\minipage{0.45\textwidth}
%				\includegraphics[width=\linewidth]{images/experiments/KP_HARRIS_0}
%				\label{fig:awesome_image1}
%			\endminipage\hfill
%			\minipage{0.45\textwidth}
%				\includegraphics[width=\linewidth]{images/experiments/KP_HARRIS_1}
%				\label{fig:awesome_image2}
%			\endminipage
%			\caption{HARRIS}
%		\end{figure}
%		\begin{figure}[!htb]
%			\minipage{0.45\textwidth}
%				\includegraphics[width=\linewidth]{images/experiments/KP_SIFT_0}
%				\label{fig:awesome_image1}
%			\endminipage\hfill
%			\minipage{0.45\textwidth}
%				\includegraphics[width=\linewidth]{images/experiments/KP_SIFT_1}
%				\label{fig:awesome_image2}
%			\endminipage
%			\caption{SIFT}
%		\end{figure}
%		\begin{figure}[!htb]
%			\minipage{0.45\textwidth}
%				\includegraphics[width=\linewidth]{images/experiments/KP_ORB_0}
%				\label{fig:awesome_image1}
%			\endminipage\hfill
%			\minipage{0.45\textwidth}
%				\includegraphics[width=\linewidth]{images/experiments/KP_ORB_1}
%				\label{fig:awesome_image2}
%			\endminipage
%			\caption{ORB}
%		\end{figure}

%\newpage

%		\begin{figure}[!htb]
%			\minipage{0.45\textwidth}
%				\includegraphics[width=\linewidth]{images/experiments/KP_HARRIS_2}
%				\label{fig:awesome_image1}
%			\endminipage\hfill
%			\minipage{0.45\textwidth}
%				\includegraphics[width=\linewidth]{images/experiments/KP_HARRIS_3}
%				\label{fig:awesome_image2}
%			\endminipage
%			\caption{HARRIS}
%		\end{figure}
%		\begin{figure}[!htb]
%			\minipage{0.45\textwidth}
%				\includegraphics[width=\linewidth]{images/experiments/KP_SIFT_2}
%				\label{fig:awesome_image1}
%			\endminipage\hfill
%			\minipage{0.45\textwidth}
%				\includegraphics[width=\linewidth]{images/experiments/KP_SIFT_3}
%				\label{fig:awesome_image2}
%			\endminipage
%			\caption{SIFT}
%		\end{figure}
%		\begin{figure}[!htb]
%			\minipage{0.45\textwidth}
%				\includegraphics[width=\linewidth]{images/experiments/KP_ORB_2}
%				\label{fig:awesome_image1}
%			\endminipage\hfill
%			\minipage{0.45\textwidth}
%				\includegraphics[width=\linewidth]{images/experiments/KP_ORB_3}
%				\label{fig:awesome_image2}
%			\endminipage
%			\caption{ORB}
%		\end{figure}

%\newpage

%		\begin{table}[H]
%			\begin{center}
%				\rowcolors{3}{}{myBlue}
%				%\begin{tabular}{l | !{\vrule width -1pt}c !{\vrule width -1pt}c | !{\vrule width -1pt}c !{\vrule width -1pt}c | !{\vrule width -1pt}c !{\vrule width -1pt}c}
%				\begin{tabular}{l | c c c | c c c}
%					& \multicolumn{3}{c|}{\textbf{Graff}} & \multicolumn{3}{c}{\textbf{Campus}}\\
%					\textbf{Algorismes} & \textbf{AB} & \textbf{A !B} & \textbf{A} & \textbf{AB} & \textbf{A !B} & \textbf{A} \\ \hline
%					Harris & - & - & - & - & - & - \\
%					SIFT & - & - & - & - & - & - \\
%					ORB & - & - & - & - & - & - \\
%				\end{tabular}
%			\end{center}
%			\caption{Detectors de keypoints - comparació}
%		\end{table}
%		\noindent
%		...


\newpage
	\subsection{Detecció i extracció de \textit{keypoints}}
		En primer lloc es prova el sistema amb imatges similars i subimatges, per comprovar  que funciona correctament en els casos més senzills.

		\begin{figure}[!htb]
			\minipage{0.24\textwidth}
				\includegraphics[width=\linewidth]{images/experiments/uni_2}
				\label{fig:awesome_image1}
			\endminipage\hfill
			\minipage{0.24\textwidth}
				\includegraphics[width=\linewidth]{images/experiments/uni}
				\label{fig:awesome_image2}
			\endminipage\hfill
			\minipage{0.24\textwidth}
				\includegraphics[width=\linewidth]{images/experiments/uni4_2}
				\label{fig:awesome_image3}
			\endminipage\hfill
			\minipage{0.24\textwidth}
				\includegraphics[width=\linewidth]{images/experiments/uni4}
				\label{fig:awesome_image3}
			\endminipage
			\caption{Imatges campus}
		\end{figure}

		\begin{table}[H]
			\begin{center}
				\rowcolors{3}{}{myBlue}
				\begin{tabular}{l | c c c c | c c c c}
					& \multicolumn{4}{c|}{\textbf{Campus 1 (subimatge)}} & \multicolumn{4}{c}{\textbf{Campus 2 (subimatge)}} \\
					\textbf{Algorismes} & \textbf{Kp1} & \textbf{Kp2} & \textbf{Parells} & \textbf{t} & \textbf{Kp1} & \textbf{Kp2} & \textbf{Parells} & \textbf{t} \\ \hline
					Harris + SIFT & 1133 & 1554 & 835 & 1.463s & 265 & 1308 & 187 & 1.070s \\
					SIFT + SIFT & 3518 & 5769 & 2784 & 2.357s & 1269 & 6168 & 1134 & 1.910s \\
					ORB + ORB & 2500 & 2500 & 793 & 0.181s & 2500 & 2500 & 243 & 0.182s \\
					ORB + BRISK & 2500 & 2500 & 1012 & 1.073s & 2500 & 2500 & 301 & 1.061s \\
				\end{tabular}
			\end{center}
			\caption{Extracció - Sub-imatges}
		\end{table}

		\begin{table}[H]
			\begin{center}
				\rowcolors{3}{}{myBlue}
				\begin{tabular}{l | c c | c c}
					& \multicolumn{2}{c|}{\textbf{Campus 1 (subimatge)}} & \multicolumn{2}{c}{\textbf{Campus 2 (subimatge)}} \\
					\textbf{Algorismes} & \textbf{Correctes} & \textbf{Erronis} & \textbf{Correctes} & \textbf{Erronis} \\ \hline
					Harris + SIFT & 816 & 19 & 184 & 3 \\
				\end{tabular}
			\end{center}
			\caption{\textit{Matching} - Sub-imatges}
		\end{table}

		\noindent
		En general, quan s'utilitzen sub-imatges el sistema funciona correctament. Es detecten bastants aparellaments i la majoria són correctes. Com es pot veure en la taula anterior,
		Harris+SIFT encerta en la majoria dels aparellaments, tot i que falla en alguns casos.\\\\
		Analitzant els casos on hi ha falsos aparellaments, podem veure que les característiques són molt similars, fallant sobretot en la finestra correcta.\\\\
		Pel que fa al temps d'execució, ORB és amb diferència el més ràpid, mentre que SIFT és el més lent.\\

		\begin{figure}[!htb]
			\minipage{0.24\textwidth}
				\includegraphics[width=\linewidth]{images/experiments/motos3}
				\label{fig:awesome_image1}
			\endminipage\hfill
			\minipage{0.24\textwidth}
				\includegraphics[width=\linewidth]{images/experiments/motos1}
				\label{fig:awesome_image2}
			\endminipage\hfill
			\minipage{0.24\textwidth}
				\includegraphics[width=\linewidth]{images/experiments/cars4}
				\label{fig:awesome_image3}
			\endminipage\hfill
			\minipage{0.24\textwidth}
				\includegraphics[width=\linewidth]{images/experiments/cars6}
				\label{fig:awesome_image3}
			\endminipage
			\caption{Imatges motos i cotxes}
		\end{figure}

		\begin{table}[H]
			\begin{center}
				\rowcolors{3}{}{myBlue}
				\begin{tabular}{l | c c c c | c c c c}
					& \multicolumn{4}{c|}{\textbf{Motos}} & \multicolumn{4}{c}{\textbf{Cotxes}} \\
					\textbf{Algorismes} & \textbf{Kp1} & \textbf{Kp2} & \textbf{Parells} & \textbf{t} & \textbf{Kp1} & \textbf{Kp2} & \textbf{Parells} & \textbf{t} \\ \hline
					Harris + SIFT & 126 & 242 & 111 & 0.162s & 135 & 136 & 102 & 0.148s \\
					SIFT + SIFT & 1195 & 1237 & 515 & 0.374s & 612 & 523 & 302 & 0.209s \\
					ORB + ORB & 2500 & 2500 & 1132 & 0.096s & 2500 & 2500 & 907 & 0.061s \\
					ORB + BRISK & 2500 & 2500 & 1250 & 0.994s & 2500 & 2500 & 1161 & 0.983s \\
				\end{tabular}
			\end{center}
			\caption{Extracció - imatges similars}
		\end{table}

		\begin{table}[H]
			\begin{center}
				\rowcolors{3}{}{myBlue}
				\begin{tabular}{l | c c | c c}
					& \multicolumn{2}{c|}{\textbf{Motos}} & \multicolumn{2}{c}{\textbf{Cotxes}} \\
					\textbf{Algorismes} & \textbf{Correctes} & \textbf{Erronis} & \textbf{Correctes} & \textbf{Erronis} \\ \hline
					Harris + SIFT & 110 & 1 & 98 & 4 \\
				\end{tabular}
			\end{center}
			\caption{\textit{Matching} - imatges similars}
		\end{table}

		\noindent
		Amb canvis d'il·luminació i \textit{blur}, els algorismes també presenten una bona taxa d'encert. ORB continua sent el més ràpid, però en aquest cas ORB+BRISK és més lent que SIFT, ja que es detecten més
		aparellaments.\\\\
		També s'han realitzat proves amb les següents fotografies, que presenten canvis de perspectiva i zoom:

		\begin{figure}[!htb]
			\minipage{0.24\textwidth}
				\includegraphics[width=\linewidth]{images/experiments/uni1}
				\label{fig:awesome_image1}
			\endminipage\hfill
			\minipage{0.24\textwidth}
				\includegraphics[width=\linewidth]{images/experiments/uni2}
				\label{fig:awesome_image2}
			\endminipage\hfill
			\minipage{0.24\textwidth}
				\includegraphics[width=\linewidth]{images/experiments/jardi_2}
				\label{fig:awesome_image3}
			\endminipage\hfill
			\minipage{0.24\textwidth}
				\includegraphics[width=\linewidth]{images/experiments/jardi2}
				\label{fig:awesome_image3}
			\endminipage
			\caption{Imatges campus i jardins}
		\end{figure}

		\begin{table}[H]
			\begin{center}
				\rowcolors{3}{}{myBlue}
				\begin{tabular}{l | c c c c | c c c c}
					& \multicolumn{4}{c|}{\textbf{Campus}} & \multicolumn{4}{c}{\textbf{Jardins}} \\
					\textbf{Algorismes} & \textbf{Kp1} & \textbf{Kp2} & \textbf{Parells} & \textbf{t} & \textbf{Kp1} & \textbf{Kp2} & \textbf{Parells} & \textbf{t} \\ \hline
					Harris + SIFT & 417 & 1099 & 71 & 1.714s & 757 & 1276 & 66 & 0.917s \\
					SIFT + SIFT & 1342 & 5921 & 150 & 2.595s & 5148 & 12582 & 120 & 4.328s \\
					ORB + ORB & 2500 & 2500 & 180 & 0.221s & 2500 & 2500 & 86 & 0.272s \\
					ORB + BRISK & 2500 & 2500 & 143 & 1.093s & 2500 & 2500 & 87 & 1.145s \\
				\end{tabular}
			\end{center}
			\caption{Extracció - canvis de perspectiva i zoom}
		\end{table}

		\begin{table}[H]
			\begin{center}
				\rowcolors{3}{}{myBlue}
				\begin{tabular}{l | c c | c c}
					& \multicolumn{2}{c|}{\textbf{Campus}} & \multicolumn{2}{c}{\textbf{Jardins}} \\
					\textbf{Algorismes} & \textbf{Correctes} & \textbf{Erronis} & \textbf{Correctes} & \textbf{Erronis} \\ \hline
					Harris + SIFT & 47 & 24 & 64 & 2 \\
					SIFT + SIFT & 94 & 56 & 91 & 29 \\
					ORB + ORB & 112 & 68 & 62 & 24 \\
					ORB + BRISK & 78 & 65 & 81 & 6 \\
				\end{tabular}
			\end{center}
			\caption{\textit{Matching} - canvis de perspectiva i zoom}
		\end{table}
		\noindent
		Quan utilitzem escenes diferents, amb canvis de zoom i de perspectiva, el nombre d'aparellaments disminueix considerablement. La taxa d'encert també disminueix, però sempre superant el 50\%.\\\\
		En les imatges de la universitat, el millor algorisme és HARRIS+SIFT (tot i que troba menys aparellaments), mentre ORB+BRISK seria el pitjor. En el cas dels jardins, tots els algorismes presenten
		una bona taxa d'encert, especialment Harris+SIFT i ORB+BRISK.\\\\
		Pel que fa als temps d'execució, SIFT és bastant més lent, encara que troba molts més \textit{keypoints}. Harris en comparació troba molts menys punts però acaba funcionant millor.\\
		
		\begin{table}[H]
			\begin{center}
				\rowcolors{3}{}{myBlue}
				\begin{tabular}{l | c c c | c c c}
					& \multicolumn{3}{c|}{\textbf{Campus}} & \multicolumn{3}{c}{\textbf{Jardins}} \\
					\textbf{Algorismes} & \textbf{RANSAC} & \textbf{OK} & \textbf{Erronis} & \textbf{RANSAC} & \textbf{OK} & \textbf{Erronis} \\ \hline
					Harris + SIFT & 30 & 30 & 0 & 24 & 24 & 0 \\
					SIFT + SIFT & 41 & 41 & 0 & 31 & 31 & 0 \\
					ORB + ORB & 60 & 60 & 0 & 33 & 33 & 0 \\
					ORB + BRISK & 42 & 42 & 0 & 41 & 41 & 0 \\
				\end{tabular}
			\end{center}
			\caption{RANSAC - canvis de perspectiva i zoom}
		\end{table}
		\noindent
		Un cop aplicat RANSAC, el nombre d'aparellaments restants es redueix un altre cop, però en aquest cas tots els aparellaments són correctes.\\\\
		També s'ha provat el sistema amb regions més concretes, que és són realment el tipus d'imatges que utilitzarà.\\

		\begin{figure}[!htb]
			\resizebox{\textwidth}{!}{%
			\includegraphics[height=1cm]{images/experiments/uni_sel}
			\includegraphics[height=1cm]{images/experiments/uni1}
			\includegraphics[height=1cm]{images/experiments/jardi_sel}
			\includegraphics[height=1cm]{images/experiments/jardi_2}}
			\caption{Imatges campus i jardins (objectes)}
		\end{figure}

		\begin{table}[H]
			\begin{center}
				\rowcolors{3}{}{myBlue}
				\begin{tabular}{l | c c c c | c c c c}
					& \multicolumn{4}{c|}{\textbf{Campus (moto)}} & \multicolumn{4}{c}{\textbf{Jardins (font)}} \\
					\textbf{Algorismes} & \textbf{Kp1} & \textbf{Kp2} & \textbf{Parells} & \textbf{t} & \textbf{Kp1} & \textbf{Kp2} & \textbf{Parells} & \textbf{t} \\ \hline
					Harris + SIFT & 67 & 417 & 30 & 0.790s & 304 & 757 & 38 & 0.375s \\
					SIFT + SIFT & 125 & 1342 & 16 & 1.058s & 570 & 5148 & 51 & 1.110s \\
					ORB + ORB & 830 & 2500 & 15 & 0.077s & 2405 & 2500 & 62 & 0.130s \\
					ORB + BRISK & 830 & 2500 & 19 & 0.940s & 2405 & 2500 & 77 & 1.037s \\
				\end{tabular}
			\end{center}
			\caption{Extracció - objectes}
		\end{table}

		\begin{table}[H]
			\begin{center}
				\rowcolors{3}{}{myBlue}
				\begin{tabular}{l | c c | c c}
					& \multicolumn{2}{c|}{\textbf{Campus (moto)}} & \multicolumn{2}{c}{\textbf{Jardins (font)}} \\
					\textbf{Algorismes} & \textbf{Correctes} & \textbf{Erronis} & \textbf{Correctes} & \textbf{Erronis} \\ \hline
					Harris + SIFT & 30 & 0 & 38 & 0 \\
					SIFT + SIFT & 15 & 1 & 49 & 2 \\
					ORB + ORB & 14 & 1 & 52 & 10 \\
					ORB + BRISK & 19 & 0 & 71 & 6 \\
				\end{tabular}
			\end{center}
			\caption{\textit{Matching} - objectes}
		\end{table}
		\noindent
		En les imatges més petites, com era d'esperar, s'obtenen menys \textit{keypoints}. Per tant el nombre total d'aparellaments trobats també serà menor.\\\\
		Harris+SIFT funciona molt bé amb aquests tipus d'imatges, essent tots els aparellaments trobats correctes. La resta d'algorismes també obtenen bons resultats,
		essent ORB el pitjor (amb un 93\% i 84\% d'encert).\\
		\begin{table}[H]
			\begin{center}
				\rowcolors{3}{}{myBlue}
				\begin{tabular}{l | c c c | c c c}
					& \multicolumn{3}{c|}{\textbf{Campus (moto)}} & \multicolumn{3}{c}{\textbf{Jardins (font)}} \\
					\textbf{Algorismes} & \textbf{RANSAC} & \textbf{OK} & \textbf{Erronis} & \textbf{RANSAC} & \textbf{OK} & \textbf{Erronis} \\ \hline
					Harris + SIFT & 18 & 18 & 0 & 23 & 23 & 0 \\
					SIFT + SIFT & 13 & 13 & 0 & 26 & 26 & 0 \\
					ORB + ORB & 12 & 12 & 0 & 33 & 33 & 0 \\
					ORB + BRISK & 18 & 18 & 0 & 49 & 49 & 0 \\
				\end{tabular}
			\end{center}
			\caption{RANSAC - objectes}
		\end{table}
		\noindent
		Un cop aplicat RANSAC, tots els aparellaments considerats \textit{inliers} són correctes.\\\\
		Finalment, s'ha provat el sistema utilitzant parells d'imatges totalment diferents, per comprovar que el nombre d'aparellaments trobats no sigui elevat.\\

		\begin{figure}[!htb]
			\resizebox{\textwidth}{!}{%
			\includegraphics[height=1cm]{images/experiments/uni_sel}
			\includegraphics[height=1cm]{images/experiments/jardi_2}
			\includegraphics[height=1cm]{images/experiments/jardi_sel}
			\includegraphics[height=1cm]{images/experiments/uni1}}
			\caption{Imatges campus i jardins (diferents)}
		\end{figure}

		\begin{table}[H]
			\begin{center}
				\rowcolors{3}{}{myBlue}
				\begin{tabular}{l | c c c c | c c c c}
					& \multicolumn{4}{c|}{\textbf{Campus 1 (subimatge)}} & \multicolumn{4}{c}{\textbf{Campus 2 (subimatge)}} \\
					\textbf{Algorismes} & \textbf{Kp1} & \textbf{Kp2} & \textbf{Parells} & \textbf{t} & \textbf{Kp1} & \textbf{Kp2} & \textbf{Parells} & \textbf{t} \\ \hline
					Harris + SIFT & 67 & 757 & 0 & 0.331s & 304 & 417 & 2 & 0.839s \\
					SIFT + SIFT & 125 & 5148 & 1 & 0.987s & 570 & 1342 & 5 & 1.113s \\
					ORB + ORB & 830 & 2500 & 2 & 0.840s & 2405 & 2500 & 7 & 0.113s \\
					ORB + BRISK & 830 & 2500 & 1 & 0.942s & 2405 & 2500 & 7 & 1.003s \\
				\end{tabular}
			\end{center}
			\caption{\textit{Matching} - imatges diferents}
		\end{table}
		\noindent
		Analitzant els resultats es pot veure que no es troben gaires aparellaments, però tot i així se'n troben alguns. Harris+SIFT sembla ser el que en troba menys.

\newpage
	\subsection{\textit{Matching} i homografia}
		Aquí podeu veure alguns dels resultats obtinguts en executar el programa seleccionant una regió més concreta:\\
		\begin{figure}[!htb]
			\resizebox{\textwidth}{!}{%
			\includegraphics[height=1cm]{images/homography}
			\includegraphics[height=1cm]{images/6}}
		\end{figure}

		\begin{figure}[!htb]
			\resizebox{\textwidth}{!}{%
			\includegraphics[height=1cm]{images/7}
			\includegraphics[height=1cm]{images/uniSel5}}
			\caption{Homografia - Resultats}
		\end{figure}
