\section{Objectius}
	L'objectiu principal del projecte consisteix a dissenyar i desenvolupar un sistema d'autolocalització per a robots mòbils.\\\\
	Aquest sistema estarà basat en tècniques de visió per computador i consistirà, bàsicament, a comparar dues imatges (una global i una altra capturada pel robot)
	i localitzar un punt o regió seleccionat per l'usuari.\\\\	
	Per arribar a aquest objectiu, es dividirà el treball en diverses fases:\\
	\begin{itemize}
		\item Estudi dels diferents algorismes de visió existents
		\item Obtenció de \textit{keypoints} en una imatge
		\item Extracció de característiques
		\item \textit{Matching} de dues imatges
	\end{itemize}
\section{Requeriments}
	El sistema d'autolocalització implementat ha de complir amb una sèrie de requeriments mínims presentats a continuació:\\
	\begin{itemize}
		\item L'usuari ha de poder seleccionar un punt o regió d'interès en una imatge donada.
		\item El sistema ha de ser capaç d'adaptar-se mínimament a diverses condicions de l'entorn (canvis de lluminositat, perspectiva, etc.).
	\end{itemize}
\section{Obstacles}
	Durant la planificació i realització del treball, s'hauran de tenir en compte els possibles obstacles que es trobaran. A continuació es detallen alguns dels problemes que es podran trobar.
	\subsection{Noves eines}
		Un dels principals obstacles serà el fet de treballar amb noves eines i algorismes. Per tal d'evitar problemes en aquest aspecte, caldrà fer una planificació acurada i documentar-se apropiadament.
		També serà important mantenir una bona comunicació amb el tutor en tot moment, per poder resoldre possibles dubtes referents als algorismes.
	\subsection{Calendari}
		Un altre obstacle important serà la falta de temps, ja que està previst realitzar el projecte en el transcurs d'un quadrimestre. Gestionar correctament el temps serà clau per aconseguir
		finalitzar el projecte sense problemes. Per tant, s'haurà de fer una planificació el més realista possible i escollir una metodologia de treball adequada i flexible.
	\subsection{Errors de programació}
		Com a qualsevol projecte on s'ha de programar, el codi serà una font important d'errors. Per això, caldrà realitzar diverses proves cada vegada que es realitzi una modificació en el codi
		o s'implementi una nova funcionalitat.
	\subsection{Condicions variables en les imatges}
		Les imatges capturades a través d'una càmera no presentaran sempre les mateixes condicions. La lluminositat, perspectiva o resolució de la imatge 
		influiran a l'hora de processar les imatges i comparar-les.\\\\
		Per intentar minimitzar aquests efectes, s'analitzaran diversos algorismes d'obtenció de punts i extracció de característiques. 
		També s'estudiarà si és necessari realitzar un preprocessament o filtratge de les imatges abans d'aplicar els algorismes.
\section{Ampliacions}
	Encara que el calendari és força estricte i no hi ha gaire marge d'ampliació, es podria estendre el projecte amb les següents ampliacions:\\
	\begin{itemize}
		\item Anàlisi del rendiment d'algorismes alternatius per l'obtenció de punts i característiques de les imatges.
		\item Creació d'una aplicació d'Android que permeti seleccionar un punt o regió d'una imatge.
		\item Execució del codi del sistema via servidor web, utilitzant les dades enviades per l'aplicació d'Android.
	\end{itemize}
\section{Metodologia}
	Per aquest projecte, s'utilitzarà una metodologia de treball àgil amb cicles de desenvolupament curts.
	Com que només hi ha un desenvolupador, no s'utilitzaran exactament les metodologies Scrum o XP\cite{Pxp} (Extreme Programming),
	però sí que s'aplicaran moltes de les pràctiques pròpies d'aquestes dues metodologies (proves, simplicitat, refacció de codi, etc.).
	Això ens donarà més flexibilitat a l'hora de fer canvis i adaptar-nos a una nova planificació.\\\\
	Es començarà treballant amb imatges de prova (casos senzills) i algorismes coneguts com ara Harris\cite{Harris} i SIFT. Més endavant, s'aniran introduint modificacions en el codi per intentar aconseguir un
	sistema capaç de funcionar amb fotografies ``reals'' i es provaran altres algorismes de visió per computador.\\\\
	Per altra banda, s'utilitzarà el mètode en cascada per la realització del curs de GEP.
\section{Eines de desenvolupament}
	El codi del projecte es desenvoluparà amb python i s'utilitzaran, sempre que sigui possible, eines de programari lliure o de codi obert.\\\\
	En cas de crear una aplicació per a dispositius Android, es realitzarà mitjançant Android Studio (Java).
	\subsection{OpenCV}
		Per tal d'utilitzar algorismes de visió per computador en el codi amb relativa facilitat, s'utilitzarà la biblioteca de codi obert OpenCV\cite{OpenCV} (Open Source Computer Vision Library),
		disponible per a python. La versió emprada serà la 3.1.\\\\
		En concret, hi haurà tres passos indispensables que faran ús d'aquesta biblioteca:\\
		\begin{itemize}
			\item {Obtenció de punts en una imatge}
			\item {Extracció de característiques}
			\item \textit{Matching} de dues imatges
		\end{itemize}
\section{Eines de seguiment}
	A continuació es detallen les eines de programari usades per fer el seguiment del treball final de grau:
	\subsubsection{LibreOffice Calc}
		Per fer un seguiment de les hores dedicades al projecte, es crearà un full de càlcul amb les hores diàries dedicades a cada tasca. S'utilitzarà LibreOffice Calc, inclòs en
		la \textit{suite} ofimàtica LibreOffice.
	\subsubsection{Gantt Project}
		Per tal d'organitzar totes les tasques a realitzar i mirar si hi ha desviacions respecte el pla inicial, s'utilitzarà l'eina de \textit{software} lliure 
		Gantt Project\cite{GanttProject}. Aquesta eina ens permetrà realitzar tant un diagrama de Gantt com un diagrama de PERT.
	\subsubsection{Git + Github}
		Tot i que no es tracta d'un projecte col·laboratiu (només hi ha un desenvolupador), s'ha decidit utilitzar el sistema de control de versions Git juntament amb la pàgina web Github. 
		D'aquesta manera es facilitarà treballar amb diverses màquines i portar un control dels canvis realitzats. A més, permetrà compartir el codi amb el director amb facilitat.
	\section{Mètode de validació}
		Es faran validacions parcials durant la realització del projecte, fent proves del sistema amb diverses imatges.
	\subsubsection{Contacte amb el director}
		Hi haurà reunions presencials amb el director, així com comunicació via correu electrònic, per tal de resoldre dubtes i validar la feina realitzada.
