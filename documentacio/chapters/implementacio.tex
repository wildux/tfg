\section{Biblioteca}
	Per facilitar la utilització del codi i possibles adaptacions o millores futures, s'ha decidit implementar una biblioteca en Python amb totes les funcions necessaries. El programa principal farà ús
	d'aquesta biblioteca, que permetrà:
	
	\begin{enumerate}
		\item{Pre-processar les imatges}
		\item{Seleccionar la regió d'interès}
		\item{Obtenir els keypoints}
		\item{Extreure les característiques}
		\item{Fer matching de característiques}
		\item{Homografia: Obtenir la regió/punt demanat}
		\item{Obtenir l'angle de rotació pel robot}
	\end{enumerate}

	\subsection{Pre-processament de les imatges}
		En primer lloc es redimensionarà la imatge per tal d'agilitzar la selecció de keypoints, extracció de característiques i matching. Desprès, s'aplicarà un filtre gaussià
		a la imatge i es retornarà la imatge en color i en escala de grisos.\\
		\begin{python}
def prep(image):
	img = cv2.resize(image, (0,0), fx=0.3, fy=0.3)
	img = cv2.GaussianBlur(img,(3,3),0)
	return img, cv2.cvtColor(img, cv2.COLOR_BGR2GRAY)
		\end{python}
	\subsection{Selecció de la regió d'interès}
		Per poder seleccionar la regió d'interès amb facilitat, s'ha definit la funció \textbf{selectRoi(imatge)}, que permet a l'usuari seleccionar una regió rectangular amb el ratolí i retalla la imatge.
		Nomes necessita la imatge com a paràmetre.\\
		\begin{python}
def click_and_crop(event, x, y, flags, param):
	global refPt, cropping, sel_rect_endpoint, img
 
	if event == cv2.EVENT_LBUTTONDOWN:	# Initial coordinates. Cropping = true
		cropping = True
		refPt = [(x, y)] 
	elif event == cv2.EVENT_LBUTTONUP:	# End coordinates. Cropping = false (done)
		cropping = False
		refPt.append((x, y)) 
		clone = img.copy()
		cv2.rectangle(clone, refPt[0], refPt[1], (0, 255, 255), 2)	# Draw a rectangle (ROI)
		cv2.imshow("image", clone)
	elif event == cv2.EVENT_MOUSEMOVE and cropping:	# Update position (moving rectangle)
		sel_rect_endpoint = [(x, y)]

def selectROI(image):
	global img, refPt, sel_rect_endpoint
	img = image ###
	cv2.namedWindow("image")
	cv2.setMouseCallback("image", click_and_crop)
	cv2.imshow('image', img)

	while True:
		if not cropping:
			sel_rect_endpoint = []
		elif cropping and sel_rect_endpoint:	# Display rectangle (moving)
			clone = img.copy()
			cv2.rectangle(clone, refPt[0], sel_rect_endpoint[0], (0, 255, 0), 1)
			cv2.imshow('image', clone)
		if (cv2.waitKey(1) & 0xFF) == ord("c"):
			break

	cv2.destroyAllWindows()
	if len(refPt) == 2:
		img = img[refPt[0][1]:refPt[1][1], refPt[0][0]:refPt[1][0]]
	return img
		\end{python}
	\subsection{Obtenció de keypoints}
		Per obtenir els punts d'interès d'una imatge, s'utilitzaràn diversos algorismes ja implementats en la biblioteca OpenCV. La funció... \\
		\begin{python}
def point_selection(gray, alg, small=False):
	kp = []

	if alg == _SIFT:
		sift = cv2.xfeatures2d.SIFT_create()
		kp = sift.detect(gray,None)

	elif alg == _HARRIS:
		gray2 = np.float32(gray)
		dst = cv2.cornerHarris(gray2,2,3,0.04)
		dst = cv2.dilate(dst,None)
		roi[dst>0.01*dst.max()]=[0,0,255]
		#SIZE = SIZE LOCAL REGION

	elif alg == _STAR:
		star = cv2.xfeatures2d.StarDetector_create()
		kp = star.detect(gray,None)

	elif alg == _ORB:
		if small:
			orb = cv2.ORB_create(nfeatures = 2500, nlevels = 8, edgeThreshold = 8, patchSize = 8, fastThreshold = 5)
		else:
			orb = cv2.ORB_create(nfeatures = 50000, nlevels = 8, edgeThreshold = 8, patchSize = 8, fastThreshold = 5)
		kp = orb.detect(gray,None)

	elif alg == _BRISK:
		brisk = cv2.BRISK_create(thresh = 8, octaves = 4)
		kp = brisk.detect(gray,None)

	elif alg == _MSER:
		mser = cv2.MSER_create()
		kp = mser.detect(gray,None)

	return kp
		\end{python}
	\subsection{Extracció de característiques}
		De la mateixa manera en que podem aconseguir keypoints, OpenCV també disposa de diversos algorismes d'extracció de característiques a partir dels keypoints d'una imatge.\\
		\begin{python}
def feature_extraction(image, kp, alg):
	des = []
	if alg == _SIFT:
		sift = cv2.xfeatures2d.SIFT_create()
		kp, des = sift.compute(image, kp)

	elif alg == _ORB:
		orb = cv2.ORB_create(nfeatures = 50000, nlevels = 8, edgeThreshold = 8, patchSize = 8, fastThreshold = 5)
		kp, des = orb.compute(image, kp)

	elif alg == _LATCH:
		latch = cv2.xfeatures2d.LATCH_create()
		kp, des = latch.compute(image, kp)

	elif alg == _BRISK:
		brisk = cv2.BRISK_create(thresh = 8, octaves = 4)
		kp, des = brisk.compute(image, kp)

	elif alg == _DAISY:
		daisy = cv2.xfeatures2d.DAISY_create()
		kp, des = daisy.compute(image, kp)

	return kp, des
		\end{python}
\newpage
	\subsection{Matching i homografia}
		En el cas dels descriptors binaris, utilitzarem la distancia Hamming, mentre que per la resta s'utilitzarà...\\
		\begin{python}
def matching(img1, img2, des1, des2, kp1, kp2, fe , rs=True):
	if fe == _LATCH or fe == _ORB or fe == _BRISK:
		bf = cv2.BFMatcher(cv2.NORM_HAMMING)
	else:
		bf = cv2.BFMatcher()

	matches = bf.knnMatch(des1, des2, k=2)
	x = -1; y = -1

	good = [] # store all the good matches as per Lowe's ratio test
	for m,n in matches:
		if m.distance < 0.8*n.distance:
			good.append(m)

	if len(good) >= MIN_MATCH_COUNT:
		src_pts = np.float32([ kp1[m.queryIdx].pt for m in good ]).reshape(-1,1,2)
		dst_pts = np.float32([ kp2[m.trainIdx].pt for m in good ]).reshape(-1,1,2)
		M, mask = cv2.findHomography(src_pts, dst_pts, cv2.RANSAC, 5.0) #If Ransac
		matchesMask = mask.ravel().tolist()
		h,w,_ = img1.shape
		pts = np.float32([ [0,0],[0,h-1],[w-1,h-1],[w-1,0] ]).reshape(-1,1,2)
		dst = cv2.perspectiveTransform(pts,M)
		img2 = cv2.polylines(img2,[np.int32(dst)],True,255,3, cv2.LINE_AA)

		x1, y1 = np.int32(dst)[0].ravel()
		x2, y2 = np.int32(dst)[1].ravel()
		x3, y3 = np.int32(dst)[2].ravel()
		x4, y4 = np.int32(dst)[3].ravel()
		x = (x1+x2+x3+x4)/4
		y = (y1+y2+y3+y4)/4
	else:
		print("Not enough matches are found", len(good),"of",MIN_MATCH_COUNT)
		matchesMask = None

	draw_params = dict(matchColor = (0,255,0), singlePointColor = None,
					matchesMask = matchesMask, flags = 2)
	return x, y, cv2.drawMatches(img1,kp1,img2,kp2,good,None,**draw_params)
		\end{python}
	\subsection{Angle de gir}
		En el cas dels descriptors binaris, utilitzarem la distancia Hamming, mentre que per la resta s'utilitzarà...\\
		\begin{python}
def getAngle(aV, w, x):
	if x == -1:
		return 0
	else:
		return (aV*x / w) - (aV/2)
		\end{python}

\section{Aplicació mòbil}
	L'aplicació mòbil d'ha desenvolupat amb Android Studio i el llenguatge de programació Java (amb l'SDK d'Android).
	\subsection{Capturar imatge}
		Per poder utilitzar la càmera del mòbil s'ha hagut d'afegir el permís necessari.\\\\
		Com que Android ja disposa d'una Activitat que ens permet llançar la càmera desde una altre aplicació, nomès ha sigut necessari fer l'Intent corresponent.\\
		\begin{python}
class MyClass(Yourclass):
  def __init__(self, my, yours):
    String = "String"
    String2 = '5 1 2 3 4'
    print String2
import numpy as np #Comment1
  # Comment2
		\end{python}
	\subsection{Selecció d'imatges de la galeria}
		Per poder utilitzar la càmera del mòbil s'ha hagut d'afegir el permís necessari.\\\\
		Com que Android ja disposa d'una Activitat que ens permet llançar la càmera desde una altre aplicació, nomès ha sigut necessari fer l'Intent corresponent.\\
		\begin{python}
class MyClass(Yourclass):
  def __init__(self, my, yours):
    String = "String"
    String2 = '5 1 2 3 4'
    print String2
import numpy as np #Comment1
  # Comment2
		\end{python}
	\subsection{Selecció de la regió d'interès}
		Per poder utilitzar la càmera del mòbil s'ha hagut d'afegir el permís necessari.\\\\
		Com que Android ja disposa d'una Activitat que ens permet llançar la càmera desde una altre aplicació, nomès ha sigut necessari fer l'Intent corresponent.\\
		\begin{python}
class MyClass(Yourclass):
  def __init__(self, my, yours):
    String = "String"
    String2 = '5 1 2 3 4'
    print String2
import numpy as np #Comment1
  # Comment2
		\end{python}
	\subsection{Enviament de dades al servidor}
		Per poder enviar dades al servidor, serà necessari l'ús d'una connexió a Internet. Per tant, s'haurà d'habilitar el permís necessari.\\\\
		Com que Android ja disposa d'una Activitat que ens permet llançar la càmera desde una altre aplicació, nomès ha sigut necessari fer l'Intent corresponent.\\
		\begin{python}
class MyClass(Yourclass):
  def __init__(self, my, yours):
    String = "String"
    String2 = '5 1 2 3 4'
    print String2
import numpy as np #Comment1
  # Comment2
		\end{python}
