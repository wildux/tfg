\label{sec:Disseny}

En aquest capítol es detalla l'arquitectura del sistema i el disseny de les aplicacions desenvolupades.
\section{Arquitectura del sistema}
	L'arquitectura del sistema està formada per tres parts diferenciades: el client (aplicació web o per \textit{smartphones}), un servidor i el robot. A continuació podeu veure un esquema de l'arquitectura i l'explicació
	de cada una de les parts.\\
	\begin{figure}[H]
		\centering
		\includegraphics[width=0.7\textwidth]{images/arquitectura}
		\captionsource{Arquitectura del sistema}{Madebyoliver i Pixel Buddha}
	\end{figure}
	\vspace{0.05cm}
	\begin{itemize}
		\item{\textbf{Aplicació mòbil/web:} Permet a l'usuari seleccionar una regió en una imatge i l'envia al servidor.}
		\item{\textbf{Servidor:} Rep la selecció de l'aplicació i les imatges del robot. S'encarrega de fer el \textit{matching} per obtenir el punt on s'haurà de desplaçar el robot.}
		\item{\textbf{Robot:} Envia les imatges capturades al servidor i espera rebre les ordres per desplaçar-se i girar.\\}
	\end{itemize}
	El treball se centrarà en la part del servidor i segons el temps disponible es realitzarà la comunicació entre el servidor i l'aplicació per a \textit{smartphones}.

\section{Aplicació de proves}
	Inicialment es desenvoluparà una aplicació per realitzar les diverses proves sense interfície gràfica, que simplement mostrarà el resultat obtingut del \textit{matching}. Més endavant, però, s'inclourà una
	interfície mínima que ens permetrà escollir les imatges a tractar i la regió d'interès (punt de destí).
	\subsubsection{Selecció de les imatges}
		Per seleccionar les imatges s'utilitzarà Tkinter\cite{Tkinter}, una GUI estàndard de Python.
		%\begin{figure}[H]
		%	\centering
		%	\includegraphics[width=0.5\textwidth]{images/fs}
		%	\caption{Selecció de les imatges}
		%\end{figure}

	\subsubsection{Selecció de la regió d'interès}
		Per poder escollir el punt de destí del robot, l'usuari haurà de seleccionar una regió d'interès en una imatge. Això es farà de manera molt senzilla, fent una selecció amb el ratolí.
		Un cop realitzada la selecció, l'usuari tindrà l'opció de refer-la (simplement seleccionant de nou) o d'acceptar-la polsant la tecla c.\\\\
		%\begin{figure}[H]
		%	\centering
		%	\includegraphics[width=0.5\textwidth]{images/selection}
		%	\caption{Selecció de la regió d'interès}
		%\end{figure}
		\begin{figure}[!htb]
			\resizebox{\textwidth}{!}{%
			\includegraphics[height=1cm]{images/fs}
			\includegraphics[height=1cm]{images/selection}}
			\caption{Aplicació de proves}
		\end{figure}
\newpage
\section{Disseny de l'aplicació mòbil}
	La interfície de l'aplicació mòbil i de l'aplicació web ha de ser molt simple. L'usuari només hauria de veure una imatge, d'on podria seleccionar-ne una regió i acceptar la selecció.\\
	%\begin{figure}[H]
	%		\centering
	%		\includegraphics[width=0.6\textwidth]{images/crop}
	%		\caption{App - Selecció de la regió d'interès}
	%\end{figure}
	\begin{figure}[H]
		\resizebox{\textwidth}{!}{%
		\includegraphics[height=1cm]{images/crop}
		\includegraphics[height=1cm]{images/webapp}}
		\caption{App/Webapp - Selecció de la regió d'interès}
	\end{figure}
	\noindent
	\\{}
	També s'hauria d'oferir una opció pels administradors, per tal de poder canviar la imatge a mostrar. El menú principal de l'aplicacio mòbil hauria de tenir les següents opcions:
	%\begin{itemize}
	%	\item{\textbf{Fer una foto}}
	%	\item{\textbf{Seleccionar una imatge de la galeria}}
	%	\item{\textbf{Opcions}}
	%\end{itemize}
	%\begin{figure}[H]
	%	\centering
	%	\includegraphics[width=0.4\textwidth]{images/menu}
	%	\caption{App - Menú}
	%\end{figure}
	\subsubsection{Fer una foto}
		Permet a l'usuari fer una foto amb la càmera del dispositiu, per després pujar-la al servidor.\\
		\begin{figure}[H]
			\centering
			\includegraphics[width=0.6\textwidth]{images/cam}
			\caption{App - Càmera}
		\end{figure}
	\subsubsection{Seleccionar una imatge}
	Si ja es disposa d'una imatge de l'entorn a la memòria del dispositiu, aquesta opció permetria a l'usuari seleccionar-la.
		\begin{figure}[H]
			\centering
			\includegraphics[width=0.4\textwidth]{images/gallery}
			\caption{App - Galeria}
		\end{figure}
	%\subsubsection{Selecció de la regió d'interès}
	%	Desprès de realitzar una captura o seleccionar una imatge del dispositiu, el programa demanarà a l'usuari que seleccioni una regió d'interès (destí) on vol que es desplaçi el robot.
	%	\begin{figure}[H]
	%		\centering
	%		\includegraphics[width=0.6\textwidth]{images/crop}
	%		\caption{App - Selecció de la regió d'interès}
	%	\end{figure}
	\subsubsection{Opcions}
		Des del menú d'opcions, l'usuari hauria de poder posar les dades per connectar-se al robot (adreça IP).\\\\
		També s'haurien de poder escollir els algorismes de visió a utilitzar.
		%I pels usuaris avançats, existeixen opcions per canviar els algorismes de visió per defecte:\\
		%\begin{itemize}
		%	\item{Obtenció de keypoints: Harris, SIFT, SURF, ORB, BRISK, MSER}
		%	\item{Extracció de característiques: SIFT, SURF, ORB, BRISK, LATCH, DAISY\\}
		%\end{itemize}
		
		%\begin{figure}[H]
		%	\centering
		%	\includegraphics[width=0.4\textwidth]{images/options}
		%	\caption{App - Opcions}
		%\end{figure}
