En aquesta secció es descriuen les tècniques de visió per ordinador i tractament d'imatges emprades durant la realització del projecte.

\section{Pre-processat digital d'imatges}
	El pre-processament digital en una imatge, consisteix en aplicar diverses tècniques per tal d'aconseguir una imatge d'on poder obtenir la informació que necessitem més facilment. Es tracta
	d'eliminar distorsions o be ressaltar determinades parts de la imatge.\\\\
	Algunes de les tècniques que es poden aplicar i s'han provat durant la realització del projecte són:

	\begin{itemize}
		\item{Suavitzat de la imatge i reducció de soroll: }
		\item{Reducció de mides: Reduir la mida de la imatge ens permet millorar el temps d'execució.}
		\item{Escala de grisos: Els píxels de la imatge passen a tenir un valor en el rang 0-255. D'aquesta manera s'aconsegueix reduir el pes de la imatge. Encara que es perd la informació del color, moltes
		vegades pot ser irellevant o fins i tot portar a errors.}
		\item{Equalització de l'histograma: Per tal de millorar el contrast de les imatges. Clahe local adaptatiu.}
		\item{Ressaltar vores: }
	\end{itemize}

	Finalment, s'ha decidit no aplicar cap filtre i utilitzar les imatges tal com són. Simplement s'ha reduït la mida i convertit la imatge a escala de grisos.

\newpage
\section{Obtenció de keypoints en una imatge}
	Consisteix en obtenir punts de la imatge amb característiques distintives, que ens puguin ser útils més endavant.
	\begin{itemize}	
		\item{Detecció de vores}
		\item{Detecció de cantonades}
		\item{Detecció de regions}
	\end{itemize}
	\begin{figure}[H]
		\centering
		\includegraphics[width=0.7\textwidth]{images/RobotKp}
		\caption{Keypoints}
	\end{figure}
	Les principals tècniques d'obtenció de keypoints utilitzades són: SIFT, HARRIS i ORB.
	També s'han provat altres algorismes com FAST, SURF, STAR, MSER o SHI-TOMASI.

	HARRIS

	SIFT

	ORB

\section{Extracció de característiques}

	L'extracció de característiques és el que ens permetrà comparar els punts obtinguts en les dues imatges.
	\begin{itemize}
		\item{Descriptors vectorials: SIFT, SURF}
		\item{Descriptors binaris: ORB, BRISK}
	\end{itemize}
	Principalment s'han utilitzat SIFT, ORB i BRISK.\\\\
	També s'han provat algorismes com SURF, LATCH o DAISY.

	SIFT

	ORB

	BRISK

\newpage
\section{Matching de característiques}

	Un cop tenim els keypoints i les característiques dels punts, necessitarem obtenir coincidencies entre els punts de les dues imatges.\\
	Bàsicament podem obtenir els matchs de dues maneres:\\
	\begin{itemize}	
		\item{Força bruta: Consisteix en provar totes les combinacions possibles per cada punt.}
		\item{Aproximació: ...}
	\end{itemize}
	Com que el temps d'execució no es una factor essencial, s'aplicarà el mètode de força bruta, una mica més lent.

	\begin{figure}[H]
		\centering
		\includegraphics[width=0.7\textwidth]{images/matching}
		\caption{Matching}
	\end{figure}

	En la imatge anterior podem veure determinats punts on el "match" és clarament erroni. Aquest error es pot minimitzar escollint punts més significatius, característiques més distinctives, aplicant el rati
	de Lowe o eliminant outliers.

\newpage
\section{Homografia}

	L'objectiu principal del programa es trobar una part d'una imatge en una altre imatge diferent i per fer això utilitzarem la homografia. Trobant la relació entre els píxels de les dues imatges podrem
	reprojectar el pla d'una imatge en l'altre i trobar el punt on volem dirigir el robot.\\\\
	A l'hora de buscar la homografia aplicarem RANSAC (Random Sample Consensus), un algorisme que ens permetrà eliminar outliers dels match trobats.\\

	\begin{figure}[H]
		\centering
		\includegraphics[width=0.7\textwidth]{images/homography}
		\caption{Homografia}
	\end{figure}
