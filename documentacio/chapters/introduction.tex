Els avenços tecnològics dels últims anys, han millorat la capacitat de les màquines per extreure informació i resoldre problemes de manera autònoma, imitant cada vegada millor
el comportament humà.\\\\
En aquest treball, es treballarà la visió per computador aplicada a un problema de robòtica.
\section{Context}
	Aquest projecte es desenvolupa com a treball final de grau dels estudis de grau en enginyeria informàtica, de l'especialitat en tecnologies de la informació.
	Es tracta d'un projecte de modalitat A, realitzat a la Facultat d'Informàtica de Barcelona (Universitat Politècnica de Catalunya) i proposat pel director Joan Climent,
	del departament d'ESAII (Enginyeria de Sistemes, Automàtica i Informàtica Industrial).
\section{Descripció del problema}
	El treball pretén resoldre un problema d'autolocalització de robots mòbils en un entorn variable, de tal manera que el robot sigui capaç de desplaçar-se d'un punt inicial a un punt final escollit per l'usuari.
	Per fer això, s'utilitzaran diverses tècniques de visió per ordinador.
\section{Actors implicats}
	En aquesta secció es descriuen els actors implicats del projecte, és a dir, totes aquelles persones que es veuran beneficiades directa o indirectament amb la realització d'aquest.\\
	\begin{itemize}
		\item \textbf{Autor/Desenvolupador:} És el màxim responsable del projecte. En tractar-se d'un treball final de grau, l'autor del projecte serà també el màxim beneficiari, ja que la realització d'aquest li permetrà acabar la carrera d'enginyeria informàtica.
		\item \textbf{Usuaris:} Qualsevol persona qui ho desitgi, tindrà accés a tots els codis desenvolupats durant el projecte, ja que es llançaran sota una llicència de programari lliure que permetrà veure i adaptar el codi a les necessitats d'altres usuaris.
		\item \textbf{Altres beneficiaris:} Qualsevol empresa o institució interessada podrà utilitzar el sistema desenvolupat i adaptar-lo a les seves necessitats, com podria ser per exemple un sistema de transport d'equipatge basat en robots.
	\end{itemize}
\section{Estat de l'art}
	\subsection{Visió per computador}
		La visió per computador\cite{Szeliski} és una ciència que té com a objectiu dotar les màquines o ordinadors de la capacitat de ``veure''.
		Es basa en l'extracció i anàlisi de dades obtingudes a partir d'imatges.\\\\
		Algunes de les aplicacions de la visió per computador són:\\
		\begin{itemize}
			\item Vehicles autònoms
			\item Realitat augmentada
			\item Reconeixement facial
			\item Restauració d'imatges
			\item Inspecció industrial 
			\item Robòtica\\
		\end{itemize}
		En aquest treball, ens interessa utilitzar la visió per computador en el camp de la robòtica, per aconseguir guiar a un robot mòbil cap a un objectiu
		determinat basant-se en la detecció d'un punt o regió en una imatge.
		\subsubsection{Nous algorismes}
			En els darrers anys, han aparegut nous algorismes d'obtenció de punts i extracció de característiques que suposen una alternativa als clàssics SIFT\cite{SIFT}
			( Scale Invariant Feature Transform) i SURF\cite{SURF} (Speeded-Up Robust Features). Alguns d'aquests algorismes són BinBoost\cite{Trzcinski13a} o un dels més recents:
			LATCH\cite{LeviHassner2016LATCH}\\\\
			En aquest projecte s'analitzarà si es adequat emprar algun d'aquests algorismes en la implementació del sistema d'autolocalització. 
	\subsection{Robòtica}
		La robòtica és un camp de la tecnologia que estudia el disseny i la construcció de robots.\\\\
		Que és, doncs, un robot? Al llarg de la historia, s'han donat diverses definicions del concepte de robot, sense existir encara una definició exacta acceptada per tothom. I a mesura que passa el temps,
		cada vegada resulta més complicat determinar si una màquina és o no un robot. Per no complicar-nos massa, entendrem com a robot una màquina programable capaç de realitzar una sèrie de
		tasques concretes interactuant amb l'entorn, ja sigui de manera automàtica o dirigida.\\\\
		Existeixen diversos tipus de robots, podent fer una classificació senzilla segons la seva arquitectura: robots mòbils, poliarticulats (industrials, mèdics, etc.), humanoides, 
		zoomòrfics\footnote{\textbf{Robots zoomòrfics:} Robots que imiten característiques pròpies de determinats animals.} i híbrids.\\\\
		Els robots mòbils, que són els que ens interessen per aquest projecte, acostumen a tenir una sèrie de sensors i dispositius per permetre'n el desplaçament, la localització, esquivar obstacles i
		realitzar tasques concretes. Alguns exemples de sensors utilitzats per robots mòbils són:\\
		\begin{itemize}
			\item Odometria: S'utilitza la informació obtinguda amb sensors de moviment (\textit{encoders} a les rodes, per exemple) per estimar la posició del robot respecte a la inicial.
			\item GPS (Global Positioning System): Es determina la ubicació del robot amb la xarxa de satèl·lits.
			\item Sensors de contacte: Permeten detectar si el robot està en contacte amb un altre objecte.
			\item Sensors d'ultrasons: Detecten objectes mitjançant ones ultrasòniques.
			\item Acceleròmetre: Determina l'acceleració del robot quan es mou. 
			\item Càmera: Permet capturar imatges de l'entorn.\\
		\end{itemize}
		En el nostre cas, només ens interessaran les dades obtingudes a través d'una càmera, és a dir, les imatges. El treball no se centrarà per tant en la part robòtica del sistema, i no es tindran en compte
		els sensors i algorismes necessaris per poder moure el robot.\\\\
		En cas d'aplicar el sistema desenvolupat en robots en un futur, aleshores s'hauran de tenir en compte altres sensors per permetre el moviment
		de la màquina i arribar a la destinació evitant obstacles.
